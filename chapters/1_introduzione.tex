%% TITOLO
\section{Introduzione}
\label{sec:introduzione}

%% TESTO
una citazione dalla bibliografia : ~\cite{Hasselmo5249}. 

Durante il corso del XX secolo le nascenti considerazioni strutturali e teoriche
affiancate da un senso comune di saturazione del sistema tonale,
hanno portato verso un cambiamento radicale di quello che è il concetto di musica.
Tali considerazioni hanno trovato fertilità in quelli
che sono stati i svariati cambi di paradigma in ambito scientifico di quel secolo.

- paradigm shift -
%% https://en.wikipedia.org/wiki/Paradigm_shift

Uno dei più importanti cambi di paradigma risiede nell'introduzione della
cibernetica e della teoria generale dei sistemi, che hanno conseguentemente
portato alla nascita del pensiero sistemico e del concetto di sistema complesso nelle scienze.
La cibernetica in particolare, è lo studio dei sistemi, o più precisamente lo studio
dell'organizzazione dei sistemi complessi.
La moderna (re)introduzione nel XX Secolo del termine cibernetica,
ebbe inizio durante gli anni della seconda guerra mondiale e si
deve al fisico matematico Norbert Weiner.
Nel 1948 Wiener pubblicò La cibernetica; in questo libro, che ottenne grande successo,
definiva l'ambito di interesse e gli obiettivi della nuova disciplina,
inaugurando anche l'uso del nuovo termine, da lui coniato.
La cibernetica ha avuto un ruolo centrale nello sviluppo di molti studi scientifici
come:
l'intelligenza artificiale, la teoria del caos, la teoria della catastrofe,
la teoria dei controlli, la teoria generale dei sistemi, la robotica, la psicologia,
ecc.
per essere più chiari,
grazie allo studio di B.Castellani e L.Gerrits possiamo visualizzare l'evoluzione
di questi paradigmi scientifici in una delle mappe più complete al riguardo, riportata
qui a seguito.


- mappa degli studi sulla complessità -
%% https://www.art-sciencefactory.com/MapLegend.html

\subsection{A subsection}
\label{sec:asubsection}

% write your text here..
\lipsum[1-2]

\begin{figure}
    \centering
    \includegraphics[width=0.25\textwidth]{figures/unito-logo.png}
    \caption{A figure}
    \label{fig:figure}
\end{figure}

\lipsum[3-4]

\begin{table}
  \caption{Sample table}
  \label{sample-table}
  \centering
  \begin{tabular}{lll}
    \toprule
    \multicolumn{2}{c}{Part}                   \\
    \cmidrule(r){1-2}
    Name     & Description     & Size ($\mu$m) \\
    \midrule
    Dendrite & Input terminal  & $\sim$100     \\
    Axon     & Output terminal & $\sim$10      \\
    Soma     & Cell body       & up to $10^6$  \\
    \bottomrule
  \end{tabular}
\end{table}

\subsubsection{A sub-subsection}

\lipsum[5]

\paragraph{A paragraph} \lipsum[6]