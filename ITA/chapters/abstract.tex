%% Abstract
\begin{abstract}
Il lavoro qui presentato è uno studio di analisi, implementazione e esecuzione
di tre Sistemi Complessi Adattivi per la performance musicale in Live Electronics.
La scelta di questi tre sistemi corrisponde a tre diversi casi di studio
nell'implementazione di dinamiche non lineari sfruttate
per la generazione dei comportamenti emergenti nei Sisitemi Complessi.
Una prima parte del lavoro tratterà dell'implementazione
e l'analisi di due brani rispettivamente di Agostino Di Scipio e Dario Sanfilippo.
Di Agostino di Scipio un sistema con non linearità provenenti dal mondo fisico,
che sfrutta fenomeni generati dalla catena elettroacustica all'interno dell'ambiente
e riportati poi all'interno del sistema digitale.
Di Dario Sanfilippo invece un sistema che sfrutta delle non linearità
appositamente programmate dal compositore
nel mondo digitale, controllate tramite agenti di autoregolazione scritti nel software.
Infine l'ultima parte del lavoro è dedicata alla composizione di un mio brano,
che sfrutta elementi di logiche ibride apprese dai due casi di studio presentati qui,
e che andrà a conclusione del lavoro di ricerca sviluppato durante il corso della tesi.
\end{abstract}
