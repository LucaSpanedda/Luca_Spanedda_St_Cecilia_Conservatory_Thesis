%% Ringraziamenti
\vfill
\LARGE \textbf{Ringraziamenti} \normalsize \newline \newline
Prima di entrare nel merito desidero esprimere la mia sincera gratitudine 
a coloro che hanno reso possibile e significativo il mio percorso.
La mia famiglia, che mi ha sempre sostenuto in ogni tipo di situazione che ho incontrato.
I miei amici, talentuosi artisti e persone di grande valore umano.
I miei compagni di corso, compositori straordinari e compagni di vita preziosi.
Giuseppe Silvi, mio maestro e amico, che mi ha accompagnato e guidato con amorevoli insegnamenti 
in questi ultimi anni mio percorso compositivo.
E insieme a lui voglio ringraziare tutti i grandi Maestri e Professori del dipartimento di 
Musica Elettronica del Conservatorio Santa Cecilia di Roma, 
che hanno contribuito alla mia crescita e formazione come compositore, 
tra cui Pasquale Citera, Nicola Bernardini, Marco Giordano, Luigi Pizzaleo, 
Federico Scalas, ed infine il Maestro Michelangelo Lupone, 
che mi ha aiutato a maturare il mio pensiero compositivo e 
mi ha sempre incoraggiato a fare del mio meglio.
Infine desidero esprimere la mia profonda gratitudine a Dario Sanfilippo e Agostino Di Scipio, 
due grandi mentori, maestri e persone straordinarie, 
che mi hanno accompagnato in questo entusiasmante percorso compositivo. 
Grazie al loro prezioso supporto ed alla loro collaborazione, 
ho potuto portare avanti le mie idee visionarie nella composizione musicale elettroacustica,
e senza di loro, questo lavoro non sarebbe stato possibile. 
Sono immensamente grato a tutti voi, 
perché senza la vostra guida non sarei stato in grado di realizzare i miei sogni, 
e questa tesi ne è una prima testimonianza della prova del mio sogno realizzato, 
che è per me solo un punto di partenza verso nuove conquiste.
Tuttavia, io credo che la bellezza dell'apprendere stia nell'essere sempre 
ad un punto di partenza e mai uno di arrivo. 
\begin{center} \textit{
It is not irritating to be where one is. 
It is only irritating to think one would like to be somewhere else.} \\
\end{center}
(John Cage - Lecture on Nothing - Incontri musicali. 
Quaderni internazionali di musica contemporanea, Agosto 1959.) \\
Grazie a tutti con affetto.