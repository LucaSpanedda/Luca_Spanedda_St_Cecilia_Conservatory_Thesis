%% TITOLO
\section{Sistemi Autonomi}
\label{sec:Sistemi Autonomi}

Nel capitolo precedente, attraverso la composizione e gli studi di Agostino Di Scipio 
nei suoi Ecosistemici Udibili, abbiamo potuto osservare
cosa si intende e come viene organizzato un sistema con non linearità provenienti dal 
mondo fisico; un sistema aperto all'ambiente esterno in riferimento 
alla cibernetica di secondo ordine.
Differentemente dall'approccio Ecosistemico, l'altro tipo di paradigma che ho trovato 
interessante studiare riguarda le particolari implementazioni di sistemi non lineari 
interamente programmati nel software, dove l'ambiente che racchiude il sistema al suo
interno e le proprie relazioni fra gli agenti, è interamente 
costruito e costituito dal DSP.
Ho voluto rintracciare questo tipo di lavoro nella composizione e gli studi
di Dario Sanfilippo: specialista di sistemi di feedback, esecutore, compositore,
i cui lavori manifestano principi di autopoiesi, evolvibilità, e di costruttivismo radicale 
nella progettazione di reti di feedback audio complesse 
che vengono implementate per performance dal vivo, destinate all'interazione 
uomo-macchina o di macchine autonome.

\begin{center} \vspace{0.5cm} \Huge - Da Continuare - \normalsize \vspace{0.5cm} \end{center}