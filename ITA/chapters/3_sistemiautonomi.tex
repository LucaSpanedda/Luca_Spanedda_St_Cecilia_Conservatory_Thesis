%% TITOLO
\section{Sistemi Autonomi}
\label{sec:Sistemi Autonomi}

Nel capitolo precedente, attraverso la composizione e gli studi di Agostino Di Scipio, 
abbiamo potuto osservare che cosa si intende, e come viene organizzato, 
un sistema con non linearità provenienti dal mondo fisico; aperto al suo ambiente esterno 
che diviene in questo caso lo spazio stesso della performance.
Sempre in riferimento alla cibernetica di secondo ordine,
abbiamo già parlato del fatto che nessun sistema è separabile e 
isolabile da un suo ambiente circostante, 
ma abbiamo osservato anche come il concetto di spazio di un sistema sia 
qualcosa di molto complesso, poiché il concetto stesso di "complessità"
sta a significare in un sistema che questo sia composto da diversi sottosistemi tipicamente in relazione fra loro.
Mentre, nel caso di Agostino Di Scipio, lo spazio del sistema che ne contiene le sue relazioni 
è costituito principalmente dal mondo fisico, 
in questo capitolo osserveremo come, per lo spazio di un sistema, 
si possa intendere anche uno spazio latente, 
che traccia le sue relazioni tra le parti in uno spazio virtuale interamente programmato nel software. 
In questo caso, l'ambiente digitale che racchiude il sistema all'interno e le sue relazioni 
tra gli agenti è interamente costituito solo dal DSP, 
inclusi i metodi per la generazione delle relative non linearità.
Ho voluto approfondire questo metodo attraverso il lavoro compositivo e gli studi di Dario Sanfilippo, 
specialista di sistemi di feedback, esecutore e compositore. 
I suoi lavori riguardano principi di autopoiesi, evolvibilità e costruttivismo radicale nella progettazione 
di reti di feedback audio complesse. \\
La modalità con cui andrò a discutere in questo capitolo la composizione di questi
sistemi autonomi, appartenenti anche questi alla più vasta categoria dei CASes (Complex Adaptive Systems), 
è ponendo un focus su un particolare lavoro di Dario Sanfilippo, Order From Noise \textit{(Homage to Heinz Von Foerster)}. \\
Order From Noise \textit{(Homage to Heinz Von Foerster)} è un progetto che implementa uno 
dei primi prototipi di Dario Sanfilippo basati sull'idea dell'\textit{adattattività dinamica}. 
Questo lavoro del compositore si basa sull'idea che i sistemi adattivi complessi, 
che sono emergenti per definizione, siano essenziali per raggiungere innovazioni formali, 
performative e tecniche, nel contesto della performance in Live Electronics e nella composizione musicale in generale. 
\footcite{sanfilippo_time-variant_2018}
In comune qui con l'opera di Agostino Di Scipio, il tipo di rapporto in cui è l'uomo ad essere totalmente a controllo della macchina,
è respinto per essere sostituito da un concetto di ibridazione: 
una condizione in cui umano e macchina cooperano per far emergere la performance. 
Nel caso delle opere di Dario Sanfilippo, tuttavia, l'estetica dei suoi lavori è interamente subordinata
al disegno del sistema stesso, e di volta in volta emerge in fenomeni musicali che sono il risultato
del tipo d'interazione che viene a crearsi fra il performer e il disegno del sistema che si era reso necessario
in partenza. \\
In alcuni casi, questa idea viene portata alle sue estreme conseguenze con opere in cui la macchina è l'unica 
ente che esegue, come accade per il brano presentato qui.

\begin{center}
    \vspace{0.5cm}
    \textit{Order from noise (2017) is based on a time-variant feedback delay network containing a 
    set of entangled nonlinear processing algorithms for audio and information signals. 
    The work is an example of autonomous self-performing system and it is realised by
    feeding the network with one millisecond of background noise from the performance
    environment. The initial recirculating noise impulse is what entirely determines the
    formal evolutions of the system which have substantially different long-term developments 
    for each different noise impulse. An approach to present the work live is that
    of reinitialising the system a number of times for a period of about three-five minutes
    to show its sensitivity to initial conditions and the long-term divergence between the
    formal structures.}\footcite{sanfilippo_time-variant_2018}
    \vspace{0.5cm}
\end{center}

\begin{center} \vspace{0.5cm} \Huge - Da Continuare - \normalsize \vspace{0.5cm} \end{center}

Lookahead Limiters per la stabilità delle Feedback Network:
In Order From Noise Dario Sanfilippo utilizza un unità per raggiungere la stabilità
globale della rete chiamata Lookahead Limiter.
<< My original design was based on two side-chained amplitude curves: 
a fast one for attack transients and one for sustained sounds, which had 
to be slow enough as to avoid intermodulation distortion. The fast curve 
was based on peak envelope estimation (see below), while the slow one was based on 
RMS to have a smooth behaviour. The curves were in a master-slave mode so that 
the effect of the fast curve would decrease with regard to the growth of the slow curve: 
that was necessary to avoid that the processed signal would be scaled down twice by both curves. 
Despite the algorithm produced good results, it needed some empirical calibrations 
and it also involved a rather large number of objects in Pure Data which would considerably load the CPU. 
Besides, I was using some objects from PD-extended and I already had in mind to migrate to Pure Data Vanilla.
A simplified design was based on only one amplitude curve calculated with a long-decay peak envelope (10 seconds). 
A peak envelope checks whether the input is greater than the output and, if that condition is true, 
it updates the peak with the new value from the input signal. 
Otherwise, the current peak decreases following an exponential curve and reaches a certain 
attenuation after a desired time which, in my specific case, is of ~60dBs. 
This would allow detecting fast attack transient and, at the same time, 
the long decay would create a smooth curve for the sustained and slow sounds. >>
https://www.dariosanfilippo.com/blog/2017/lookahead-limiting-in-pure-data/

This was a good compromise for my systems, although such a long decay is not always desirable: occasionally, 
that would create silences after high-amplitude impulses for the attenuating curve 
would decrease very slowly and the input signal would be scaled down even if below the limiting threshold.