%% TITOLO
\section{Introduzione}
\label{sec:Introduzione}

\begin{center}

\vspace{0.5cm}

\textit{I suppose I could mention from one of the very earliest computer scientists, 
whose name was Norbert Wiener, and he wrote a book back in the ’50s, 
from before I was even born, called “The Human Use of Human Beings.”...
And he has this amazing line where he says, one could imagine, as a thought experiment 
— and I’m paraphrasing, this isn’t a quote — 
one could imagine a global computer system where everybody has devices on them all the time, 
and the devices are giving them feedback based on what they did, 
and the whole population is subject to a degree of behavior modification. 
And such a society would be insane, could not survive, could not face its problems.
And then he says, but this is only a thought experiment, 
and such a future is technologically infeasible.
And yet, of course, it’s what we have created, 
and it’s what we must undo if we are to survive.}

\\

Jaron Lanier: How we need to remake the internet, TED Talk 2018

\vspace{0.5cm}

\end{center}

All'inizio del XX Secolo si è manifestata una situazione globale 
di importanti cambiamenti in tutti gli ambiti.
Nel mondo dell'arte con la nascita delle avanguardie artistiche e
in risposta alla problematica del dover trovare nuovi modi di fare
arte ,che riflettano le problematiche della società attuale.
Nelle scienze, con l'esigenza di dover introdurre nuovi paradigmi
per far fronte alla grande crisi dei fondamenti e delle certezze.
Durante il corso del secolo poi, questi cambiamenti hanno portato 
ad importanti punti di incontro non più occasionali fra questi due ambiti. 
La figura dell'artista inizia ad interessarsi alle nuove tecnologie
e teorie scientifiche, e questo si riflette in tutte le avanguardie.
Nel caso di questa tesi invece, l'attenzione che ripongo in questo scenario 
è su un particolare cambio di paradigma scientifico
nascente durante la seconda guerra mondiale e consolidato
dopo il termine di questa,
verso gli anni '50 con la nascita della cibernetica, 
ed è quello delle scienze della complessità.
La complessità, non è proprietà delle cose in sé stesse o un ambito particolare
di una sola scienza che tratta di questo, 
ma più un nuovo modo di pensare e di osservare i fenomeni 
della realtà.
Se vogliamo, l'idea primordiale di comportamento complesso è rintracciabile fino
nelle più antiche società, dove il principio del mondo non è opera di una divinità creatrice,
ma ogni cosa deriva invece dalla materia e dall’evoluzione del caos primordiale,
come è nel caso del mito greco nelle varie Teogonie, fra cui la più famosa quella scritta da Esiodo.
Nelle mitologie antiche dunque il caos è quasi sempre
contrapposto al cosmo, nel senso di universo disordinato il primo e ordinato
il secondo, e di disordine il caos rimarrà sinonimo nell'ambito delle scienze fino alla fine del XIX secolo, quando nell'ambito della meccanica classica Henri Poincaré osservò la possibilità di un comportamento fortemente irregolare di alcuni sistemi dinamici studiando il problema dei tre corpi. 
La scoperta di Poincaré segnerà un punto di svolta che verrà ripreso poi solamente negli anni '50 del secolo successivo dal meteorologo Edward Norton Lorenz.
Nel 1963 Lorenz pubblica il suo articolo Deterministic Nonperiodic Flow, nel quale tratta del comportamento caotico in un sistema semplice e deterministico, con la formazione di un attrattore strano,
aprendo la strada quella che diverrà la Teoria del Caos, e mostrando come in realtà all'interno dell'ordine ci siano forme di disordine, e all'interno del disordine ci siano forme di ordine.

\subsection{La Cibernetica}
\label{sec:La Cibernetica}
In questo complesso scenario del XX secolo, 
uno dei più importanti avanzamenti nelle scienze
al termine della seconda guerra mondiale risedette nell'introduzione della
cibernetica e della teoria generale dei sistemi, che hanno conseguentemente
portato per prime alla nascita del pensiero sistemico e del concetto
di scienze della complessità. \\
La cibernetica è la scienza che studia i principi astratti di organizzazione
nei sistemi complessi, ed ebbe inizio durante gli anni della seconda guerra
mondiale, merito del fisico e matematico Norbert Weiner.
Nel 1940 Wiener insieme ad altre ad altre prominenti figure provenienti
da diversi ambiti scientifici,
come Ross Ashby, Margaret Mead, Gregory Bateson, Heinz von Foerster,
partecipano ad una serie di conferenze
multidisciplinari chiamate "The Macy Conferences", inizialmente intitolate come
"Feedback Mechanism in Biology and the Social Sciences"
con l'obiettivo comune di andare a definire
gli ambiti di interesse della nuova scienza.
Il concetto sviluppato dai greci - kybernetes -
venne poi ripreso da Norbert Weiner nel 1948,
che ispirato dalla meccanica ed i suoi risultati durante la guerra
e contemporaneamente dallo sviluppo della teoria della comunicazione
(o informazione) di Claude Shannon,
e con la volontà di sviluppare una teoria generalizzata dei principi di
organizzazione e controllo nei sistemi emersi durante le conferenze,
pubblicherà un libro nel 1948:
La cibernetica, controllo e comunicazione nell'animale e nella macchina;
in cui definiva l'ambito di interesse e gli obiettivi della nuova disciplina
inaugurando anche l'uso del nuovo termine da lui coniato.
A seguito di questo libro che riscuoterà
un importante successo, le conferenze presero il nome di
"Cybernetics, Circular Causal, and Feedback Mechanism in Biological and Social Systems",
riconoscendo Wiener come la principale figura di spicco della nuova scienza. \\
In particolare come evidenziato fino ad ora dalla sua natura multidisciplinare,
la cibernetica non si interessa di individuare in cosa consistano questi sistemi,
ma più che altro comprenderne il loro funzionamento:
\\ \\
- come usano l'informazione e come la scambiano gli agenti
\\
- come collaborano fra loro in direzione di un obiettivo comune
\\
- come contrastano il rumore nel trasferimento dell'informazione
\\
  e così via...
\\ \\
Le fortunate premesse iniziali della cibernetica risiedevano in una convinzione
da parte di questi scienziati provenienti dai differenti ambiti disciplinari,
che esistesse uno "schema processuale" comune ad organismi viventi e macchine,
rintracciato attraverso una ricerca uniforme garantita da dell'utilizzo di un metodo
"sintetico" e "comportamentale".
Fra gli anni '60 e la metà del '70, grazie agli scienziati
Heinz von Foerster, Margaret Mead, e altri,
si compierà un ulteriore passo fondamentale che porterà
il pensiero sistemico verso il consolidamento in una scienza più concreta
dando vita alla "Cibernetica di secondo ordine",
anche chiamata come "la cibernetica dei sistemi di osservazione",
che consiste nell'applicazione
ricorsiva della cibernetica a se stessa e la pratica riflessiva della cibernetica
secondo tale critica.
La differenza fra cibernetica di primo e secondo ordine risiese nel fatto,
che mentre nel primo periodo lo studioso di cibernetica (di primo ordine)
studiava un sistema da un punto di vista passivo, da quello dell'osservatore
dei comportamenti di un sistema.
Il  cibernetico di secondo ordine lavora ed interviene nel comportamento
e nella costruzione di un sistema complesso,
riconoscendo il sistema come un agente con cui interagire e
riconoscendo esso stesso come agente nell'interazione col sistema. \\
A partire dalle sue importanti premesse,
la cibernetica ha conseguentemente poi avuto un ruolo centrale nello sviluppo di
molti studi scientifici e la nascita
di nuovi ambiti come: l'intelligenza artificiale, la teoria del caos,
la teoria della catastrofe,
la teoria dei controlli, la teoria generale dei sistemi, la robotica,
la psicologia, le scienze sociali, e così via.

\subsection{Le cibernetiche nella musica}
\label{sec:Le cibernetiche nella musica}
All'inizio degli anni '60 in seno alle nascita delle scienze complesse,
l'uso di sistemi di feedback e la rilevanza dei circuiti informativi chiusi
nelle strutture organizzate,
ha goduto di uno slancio popolare anche nel mondo della musica
e più in generale dell'arte.
Uno dei primi nella storia dell'arte ad evocare l'uso della cibernetica
nei propri lavori è stato
Nicolas Schoeffer con il suo ciclo di lavori "spazio-dinamici",
in particolare ha creato la prima installazione ad implementare meccanismi
di auto-regolazione, il CYSP-1 \footcite{SanfilippoValle},
capace di essere sensibile all'ambiente esterno e a se stesso
grazie ad una serie di tecnologie offerte dalla compagnia Philips (fotocellule e microfoni),
e reagire sonoramente a questi stimoli riproducendo
una serie di registrazioni composte dal compositore francese Pierre Henry,
collaboratore di Pierre Schaeffer ed insieme a lui figura centrale nella nascita della Musique concrète.
Un'altra importante esperienza del periodo iniziale è quella del compositore Roland Kayn,
che dopo essersi avvicinato alla musica elettronica sotto la guida di Herbert Eimert
nello studio di Colonia (1954),
e dopo essersi trasferito a Roma nel 1960, dal 1964 assieme ad Aldo Clementi e Franco Evangelisti
fonda il Gruppo di improvvisazione Nuova Consonanza, \footcite{Kaynbio}
del quale fece parte sino al 1968,
ed in quel periodo ispirato dalle teorie della cibernetica iniziò a sperimentare
estensivamente con sistemi di autoregolazione basati su feedback loops,
sia come modelli formali per composizioni strumentali che come reti di generatori di segnale analogici.
Tuttavia, a parte casi popolari di deliberate dichiarazioni
formali da parte degli artisti,
come nel caso di Roland Kayn,
non bisogna pensare a questi lavori appena citati (ed altri riportati a seguito),
come ad atti pioneristici che sancisono la nascita della cibernetica in musica,
ma proprio come si dice per la scoperta del fuoco
lo scenario più accurato risiede probabilmente nel fatto che
tanti autori provenienti da diverse parti del mondo, nello stesso periodo
sono stati influenzati e si sono influenzati a vicenda con le stesse idee
provenienti da un interesse condiviso per le teorie cibernetiche di Weiner e delle Macy Conferences.
Si può pensare ad esempio a quelle che sono le esperienze dello studio di Colonia:
nel 1951, Herbert Eimert e Werner Meyer-Eppler persuasero il direttore della NWDR, Hanns Hartmann,
a creare uno Studio per la Musica Elettronica, che Eimert diresse fino al 1962.
Questo è diventato lo studio più influente al mondo durante gli anni 1950 e 1960,
con ospiti alcuni dei più importanti compositori contemporanei provenienti da tutta europa,
come il già citato Roland Kayn, Franco Evangelisti, Karlheinz Stockhausen, Herbert Brun,
Cornelius Cardew, e molti altri. \footcite{EMSColonia}
Non a caso in quel periodo il lavoro di ricerca condotto da Werner-Meyer Eppler,
scienziato, musicista ideatore e direttore dello studio di Colonia,
ha trovato sin dalla nascita dello studio fondamento in quelle che sono state
le teorizzazioni della Cibernetica.
C'è poi il caso di Franco Evangelisti,
come citato prima fondatore insieme a Roland Kayn del Gruppo Improvvisazione Nuova Consonanza,
che in quel periodo (qualche anno prima della fondazione del Gruppo a Roma)
si trova nello studio di Colonia per lavorare al brano "Incontri di Fasce Sonore",
e quando farà ritorno a Roma poi citerà più volte deliberatamente in interviste, scritti,
e altre documentazioni, il suo approccio sistemico/cibernetico a quelle che saranno le esperienze con il Gruppo.
Se cambiamo paese e passiamo dall'Europa od osservare l'America in quel periodo,
possiamo pensare a quelli che sono i lavori di Louis e Bebe Barron,
con i circuiti in retroazione destinati al corto circuito
e utilizzati appositamente come materiale per la generazione acustica di trame incise su nastro,
o ai lavori pioneristici di
John Cage, David Tudor, Robert Ashley e Steve Reich,
che sfruttano ed esplorano l'effetto Larsen in modo artistico insieme
alle logiche di Feedback. \\
Un secondo periodo costituito da un approccio sistemico più consapevole
che inizia a tracciare la strada per un pensiero ecosistemico della composizione,
inizia invece dal lavoro
di Alvin Lucier, che nel 1969 scriverà quello che sarà un brano emblematico per
la cibernetica in musica "I'm sitting in a room",
è un altro brano importante per quelle che sono
le logiche di interazione sistemiche fra uomo/macchina/ambiente
e che sancisce una volta per tutte
l'interazione sistemica dove il musicista l'ambiente e lo strumento sono parti di un insieme del sistema "più complesso" con un comportamento collettivo derivato dai singoli agenti,
in un un'interazione con l'ambiente circostante.
In I'm sitting in a room, un performer al centro della stanza
recita in un microfono un testo che descrive il fenomeno che avverrà poco a poco,
la voce recitante nel microfono viene registrata e poi riprodotta da altoparlanti
posti nella stanza, il suono della regitrazione riprodotta da questi altoparlanti
viene registrato nuovamente durante la riproduzione, l'operazione
viene ripetuta in un in una casualità circolare di volta in volta dove alla fine rimarranno
solo i contributi provenienti dalla stanza, dalla voce e dalla catena elettroacustica,
dando vita nel loro insieme ad un effetto Larsen, la natura
nonlineare del processo e degli agenti porterà di volta in volta ad un risutato
sempre differente.
Dopo l'esperienza di Lucier, nel 1974 Nicolas Collins compone "pea soup"
mentre è studente alla Wesleyan University.
Pea soup consiste in una rete adattiva di circuiti analogici (3 Countryman Phase Shifters),
che intona il feedback positivo dell'effetto Larsen ad una frequenza risonante diversa
ogni volta che questo inzia ad emergere.
Ad oggi svariati compositori a partire dalle trame delineate dalle scienze complesse e
dai lavori citati operano nell'ambito della musica elettronica con un approccio sistemico,
un importante caso è quello di Agostino Di Scipio, che contribuisce significatamente
nell'ambito della computer music sin dai primi anni '90, divenendo una
delle figure più importanti nell'area della composizione ecosistemica e nel suo
caso in particolare del live electronics, o quello di Dario Sanfilippo
compositore e ricercatore con all'attivo recenti importanti pubblicazioni e lavori
nell'ambito dei sistemi di feedback in musica, e in particolare di non linearità nei sistemi
in DSP.

\subsection{Il Feedback}
\label{sec:Il Feedback}
Il feedback (o retroazione) è un concetto cibernetico che sta ad indicare
la capacità di un sistema di autoregolarsi tenendo conto degli effetti scaturiti
dalla modificazione delle caratteristiche del sistema stesso.
In termini appartenenti alla fisica, è la capacità di un sistema dinamico
di tenere conto dei risultati del sistema per
modificare le caratteristiche del sistema stesso.
Negli esseri viventi, ad esempio, i sistemi a retroazione negativa e positiva
sono ampiamente utilizzati per regolare l'omeostasi dell'organismo.
Esistono idealmente due tipologie di Feedback:
\\ \\
- a Retroazione Positiva
\\
- a Retroazione Negativa
\\ \\
La retroazione positiva tende ad accelerare un processo,
mentre la retroazione negativa a rallentarlo.
La retroazione negativa aiuta a mantenere la stabilità di un sistema,
contrastando i cambiamenti provenienti dall'ambiente esterno.
Mentre la positiva tende generalmente alla complessità.
Nel controllo di un sistema complesso,
come può essere ad esempio quello del feedback acustico,
introdurre delle linearità tramite retroazione
vuole dire costringere la complessità
a dei comportamenti prevedibili, si può pensare ad esempio
all'intonazione del feedback,
che da un comportamento complesso della sorgente e del ricettore arriva
ad uno lineare.
Mentre introdurre delle non-linearità nel sistema tramite la retroazione,
vuoldire portare questo verso comportamenti non più prevedibili.
Questi due tipi di comportamento possono essere ottenuti per l'appunto
sia velocizzando che rallentando questi processi,
in maniera dipendente dal caso specifico.
I filtri digitali o analogici nell'audio,
possono essere pensati per esempio come uno strumento
di contrasto rispetto a questo tipo di comportamenti:
dove se si allineano le fasi si creano dei poli,
mentre se si disallineano si punta alla complessità del sistema.
Di fatto la storia delle tecnologie elettroacustiche ha in generale
da sempre incorporato il principio
del feedback sin dalle sue origini, basti pensare a tecnologie
come la valvola audion di Lee De Forest (1910),
o i circuiti di feedback negativo di Harold Black (1920) [\cite{echodiscipio}].
Fino al caso del feedback elettroacustico (effetto Larsen), che abbiamo visto
esser stata una delle risorse centrali dei primi compositori cibernetici.
Addentriamoci ora verso una spiegazione più tecnica del feedback acustico
che ci servirà per comprendere più
a fondo la poetica di questi compositori,
parafrasando l'articolo appena esposto di Agostino Di Scipio:
In una stanza vengono collegati fra loro (tramite uno o più stadi di amplificazione) gli
elementi di una catena elettroacustica molto elementare: Microfono ed Altoparlante. Anche in
una situazione di "silenzio" ideale, viene catturata dal microfono in ogni caso,
inevitabilmente, perfino la turbolenza minima appena udibile presente nel rumore ambientale.
Questo suono catturato dal microfono, viene amplificato e riprodotto dall’altoparlante a sua
volta. E se l'amplificazione è sufficiente il suono dall’altoparlante ritorna al microfono e
il design della catena elettroacustica si chiude su se stessa,
creando un circuito di
retroazione (loop), anche detto di feedback.
Il livello di ampiezza, le caratteristiche
tecniche trasduttive di microfono e altoparlante, la loro distanza relativa, la distanza dalle
pareti ed altri potenziali fattori influenti, delineano un’oscillazionein fase con il segnale
che viene a sommarsi ad esso e viene amplificata e riprodotta a sua volta con ampiezza via via
crescente, idealmente illimitata. Con livelli di guadagno non troppo elevati, ciò che si
genera è un fastidio udibile, una sorta di 'alone':
la reiniezione del suono decade più o meno
rapidamente, in una specie di effetto riverbero composto da suono spettralmente irregolare.
Con livelli di guadagno più elevati, il feedback loop entra in un regime di auto-oscillazione.
A causa della reiniezione ripetuta, il rumore di fondo appena udibile ma spettralmente ampio
si accumula nel loop e alla fine (rapidamente) produce un suono sostenuto sempre più forte di
uno spettro più ristretto - questo è spesso sentito come un tono di picco di altezza definita
o un gruppo di toni. E questo è l'effetto Larsen: (dal nome del fisico Søren Absalon Larsen
che per primo ne scoprì il principio), detto anche feedback acustico o più prosaicamente
ritorno, è timbricamente riconoscibile come un tipico fischio stridente, che si sviluppa come
abbiamo detto quando i suoni emessi da un altoparlante vengono captati con sufficiente
"potenza di innesco" da un trasduttore (che può però anche essere oltre al microfono, un
pick-up di uno strumento musicale elettrico, come una chitarra o un basso, o un trasduttore di
altra natura).