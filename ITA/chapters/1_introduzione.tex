%% TITOLO
\section{Introduzione}
\label{sec:Introduzione}

All'inizio del XX Secolo si manifestò una situazione globale
di importanti cambiamenti in tutti gli ambiti. \\
Nel mondo dell'arte con la nascita delle avanguardie artistiche e
in risposta alla problematica del dover trovare nuovi modi di fare
che riflettano le tematiche della società attuale.
Nelle scienze, con l'esigenza di dover introdurre nuovi paradigmi
per far fronte alla grande crisi dei fondamenti e delle certezze.
Durante il corso del secolo poi, questi cambiamenti hanno portato
ad importanti punti di incontro non più occasionali fra questi due ambiti.
La figura dell'artista inizia ad interessarsi alle nuove tecnologie
e teorie scientifiche, e questo è stato un importante punto in comune a tutte
le avanguardie dell'epoca, inclusa quella musicale. \\
Nel caso di questa tesi, l'attenzione che ripongo a questo scenario del XX Secolo
è in particolare su un cambio di paradigma scientifico
nascente durante la seconda guerra mondiale e consolidato al termine di questa,
la nascita della cibernetica e la conseguente formulazione
di scienze della complessità.
La complessità, non è un ambito trattato da una sola scienza,
ma più un nuovo modo di pensare e di osservare i fenomeni
della realtà. \\
Se nella cosmologia greca, questa realtà veniva spiegata come caos per 
l’insieme disordinato e indeterminato degli elementi materiali, contrapposto al cosmos che rappresenta l'ordine,
come si può osservare fra i miti delle varie Teogonie
fra cui la più famosa quella scritta da Esiodo. \footcite{esiodoteogonia}
Oggi la parola caos ha invece un significato decisamente meno generale. 
Il caos, anzi il caos deterministico, è la scienza che studia i 
sistemi dinamici che esibiscono una sensibilità esponenziale rispetto alle condizioni iniziali.
O, in termini più rigorosi, è la scienza che studia la dinamica dei sistemi non lineari.
Questo cambio di paradigma ha avuto inizio verso la fine del XIX secolo,
quando uno studioso nell'ambito della meccanica classica, Henri Poincaré,
osservò e analizzò la possibilità di un comportamento fortemente irregolare
di alcuni sistemi dinamici studiando il problema dei tre corpi,
che lo portò alla scoperta del caos matematico. \footcite{poincaréproblema}
La scoperta di Poincaré segnerà un punto di svolta che verrà
ripreso poi solamente negli anni '50 del secolo successivo dal meteorologo
Edward Norton Lorenz,
quando nel '63 Lorenz pubblicherà il suo articolo 
\textit{Deterministic Nonperiodic Flow} \footcite{Lorenzdnf},
nel quale tratta del comportamento caotico in un sistema semplice
e deterministico, con la formazione di un attrattore strano,
e aprendo di fatto ufficialmente la strada quella che diverrà poi
la Teoria del Caos,
mostrando come in realtà all'interno dell'ordine emergano forme di disordine,
e all'interno del disordine siano presenti forme di ordine.

\subsection{La Cibernetica}
\label{sec:La Cibernetica}
Tornando a questo complesso scenario del XX secolo,
al termine della seconda guerra mondiale e
qualche decennio prima della formulazione della Teoria del Caos,
uno dei più importanti avanzamenti nelle scienze che contribuì alla
formazione del primo paradigma della complessità,
risiedette nell'introduzione della cibernetica. \\
La cibernetica è la scienza che studia i principi astratti di organizzazione
nei sistemi complessi, ed ebbe inizio durante gli anni della seconda guerra
mondiale, merito del fisico e matematico Norbert Wiener.
Nel '40 Wiener insieme ad altre ad altre prominenti figure provenienti
da diversi ambiti scientifici,
come Ross Ashby, Margaret Mead, Gregory Bateson, Heinz von Foerster,
partecipano ad una serie di conferenze
multidisciplinari chiamate "The Macy Conferences", inizialmente intitolate come
"Feedback Mechanism in Biology and the Social Sciences"
con l'obiettivo comune di andare a definire
gli ambiti di interesse della nuova scienza.
A seguito nel '48,
ispirato dalla meccanica ed i suoi risultati conseguiti durante la guerra
e contemporaneamente dallo sviluppo della teoria della comunicazione
(o informazione) di Claude Shannon,
con la volontà di sviluppare una teoria generalizzata dei principi di
organizzazione e controllo nei sistemi emersi durante le conferenze,
pubblicherà un libro:
La cibernetica, controllo e comunicazione nell'animale e nella macchina;
in cui definiva l'ambito di interesse e gli obiettivi della nuova disciplina
inaugurando anche l'uso del nuovo termine da lui coniato.
A seguito di questo libro che riscuoterà
un importante successo, le conferenze presero il nome di
"Cybernetics, Circular Causal, and Feedback Mechanism
in Biological and Social Systems", \footcite{fabbrisgiustinianocyb}
riconoscendo Wiener come la principale figura di spicco della nuova scienza. \\
In particolare come evidenziato fino ad ora dalla sua natura multidisciplinare,
la cibernetica non si interessa di individuare in
cosa consistano questi sistemi,
ma più che altro comprenderne il loro funzionamento. \\
Le fortunate premesse iniziali della cibernetica risiedevano in una convinzione
da parte di questi scienziati provenienti dai differenti ambiti disciplinari,
che esistesse uno "schema processuale" comune ad organismi viventi e macchine,
rintracciato attraverso una ricerca uniforme
garantita dall'utilizzo di un metodo
"sintetico" e "comportamentale". 
L'aspetto meta-disciplinare del pensiero cibernetico,
esplicito nella sua fondazione, godrà però di una fama più popolare 
che conseguirà in importanti realizzazioni 
tra l'inizio degli anni '60 e la metà del '70, 
grazie al contributo degli scienziati
Heinz von Foerster, Margaret Mead, Gregory Bateson, e altri;
proprio in quel periodo si compieranno ulteriori passi fondamentali che porteranno
il pensiero sistemico verso un consolidamento in una scienza più concreta,
uno dei casi più rilevanti in questo senso è il passaggio da "Cibernetica di primo ordine"
a "Cibernetica di secondo ordine", anche chiamata come "la cibernetica dei sistemi di osservazione",
questa distinzione si deve al fisico e filosofo Heinz Von Foerster \footcite{scottsecondordercyb}
La differenza fra cibernetica di primo e secondo ordine risiede principalmente nel fatto,
che mentre nel primo periodo lo studioso di cibernetica (di primo ordine)
studiava un sistema da un punto di vista passivo, da quello dell'osservatore
dei comportamenti del un sistema, senza tener conto della propria influenza 
nei casi di studio.
Il  cibernetico di secondo ordine invece, è consapevole della propria 
influenza all'interno del sistema, riconoscendo il sistema come un agente con cui interagire e
riconoscendo esso stesso come agente nell'interazione col sistema, 
e di conseguenza proprio come spiega Heinz Von Foerster nei suoi scritti,
di come la realtà per come la percepiamo possa esistere
soltanto se si tiene conto del fatto che la sua rappresentazione è fortemente dipendente 
dall'esistenza di un ambiente che racchiude il sistema stesso, \footcite{understandingunderstanding}
l'organizzazione ed eventuale disorganizzazione di nessun sistema può esistere 
se questo comportamento non viene studiato tenendo in considerazione l'ambiente che lo racchiude.
A partire da queste importanti premesse e dalle conseguenze 
che la cibernetica ha avuto in tutti gli ambiti delle scienze, è doveroso 
notare che ha avuto un ruolo centrale nello sviluppo di
molti studi scientifici e la nascita
di nuovi ambiti come: l'intelligenza artificiale, la teoria del caos,
la teoria della catastrofe,
la teoria dei controlli, la teoria generale dei sistemi, la robotica,
la psicologia, le scienze sociali, e così via.

\subsection{Le cibernetiche nella musica}
\label{sec:Le cibernetiche nella musica}
All'inizio degli anni '60 in seno alle nascita delle scienze complesse,
l'uso di sistemi di feedback e la rilevanza dei circuiti informativi chiusi
nelle strutture organizzate
ha goduto di uno slancio popolare anche nel mondo della musica
e più in generale dell'arte. \\
Tuttavia come vedremo, a parte casi popolari di deliberate dichiarazioni
formali da parte degli artisti, non bisogna pensare ai lavori che andremo a citare
come atti pionieristici che sanciscono una volta per tutte la nascita della cibernetica in musica,
ma è più corretto pensare alle questioni sistemiche come ad una sensibilità
comune condivisa in un certo periodo da diversi autori provenienti da diverse parti del mondo,
che sono stati influenzati e si sono influenzati a vicenda
con le stesse idee per un interesse condiviso riguardo
le teorie cibernetiche di Weiner e delle Macy Conferences. \\
% Europa - Roma
In Europa nel '51, Herbert Eimert, Werner Meyer-Eppler e Robert Beyer
di comune accordo con il direttore della NWDR, Hanns Hartmann,
attrezzarono e crearono quello che divenne il primo studio per la Musica Elettronica. 
\footcite{discipiocircuitideltempo}
Questo divenne lo studio più influente al mondo durante gli anni '50 e '60,
con ospiti alcuni dei più importanti compositori contemporanei provenienti da tutta Europa,
come Franco Evangelisti, Karlheinz Stockhausen, György Ligeti, Roland Kayn, Herbert Brün,
Cornelius Cardew, e molti altri.
In quel periodo il lavoro di ricerca condotto da Werner-Meyer Eppler,
scienziato, musicista, fra gli ideatori e direttori dello studio di Colonia e
direttore dell'istituto di fonetica all'università di Bonn,
pone una certa attenzione in quelle che furono le prime
teorizzazioni della teoria dell'informazione e della cibernetica,
che porteranno l'autore alla scrittura di importanti testi di ricerca.
Fra le altre cose Werner-Meyer Eppler nel '50 escogita un uso particolare
del magnetofono, mediante retroazione \textit{(Feedback)} fra le testine di riproduzione e 
registrazione escogita una forma di \textit{Tape Delay}. \footcite{discipiocircuitideltempo}
Dalle esperienze dello studio di Colonia ne usciranno molti compositori interessati
alle teorie cibernetiche. \\
Un caso importante in questo scenario è quello del compositore Roland Kayn,
Il progetto di musica cibernetica di Kayn ha ricevuto il suo impulso iniziale
quando nel '53, ad allora giovane musicista e studente universitario,
venne in contatto con il filosofo Max Bense professore all'Università Tecnica di Stoccarda.
Subito dopo il suo primo incontro con Bense sempre nel '53, Kayn entrò in contatto con Herbert
Eimert presso lo studio elettronico della Westdeutscher Rundfunk di Colonia.
Roland Kayn era affascinato dal potenziale sonoro offerto dalle nuove tecnologie,
ma trovava che l'estetica serialista dominante nello studio in quegli anni
era per lui qualcosa di troppo restrittivo, esperienza che lo portò
per i successivi dieci anni a concentrarsi
principalmente sulla composizione strumentale e le applicazioni delle teorie
cibernetiche in modo formale. \footcite{thomaswpattesonkayn}
Sempre in quel periodo, sostanzialmente diverso e altrettanto importante
è il caso di Franco Evangelisti, che dopo essersi avvicinato alla musica elettronica
anche lui sotto la guida e su invito di Herbert Eimert allo studio elettronico
della Westdeutscher Rundfunk di Colonia nel '56,
iniziò le sue ricerche che dopo un anno e mezzo di intenso lavoro lo portarono
al completamento della sua forse più importante composizione elettronica,
Incontri di fasce sonore, trasmessa da Radio Colonia nel '57.
Nel '59, Evangelisti è nuovamente in Italia, dove fu tra i promotori
della Settimana Internazionale di Nuova Musica a Palermo.
L'anno seguente, assieme ad altri musicisti
fondò l'Associazione Nuova Consonanza, con lo scopo di diffondere
"la musica contemporanea italiana e straniera con concerti convegni ed eventi di vario tipo.
Dall'associazione nacque più tardi l'omonimo Gruppo di improvvisazione
che allora veniva presentato come "il primo ed unico gruppo formato da compositori-esecutori"
e che permise ad Evangelisti di mettere in pratica le proprie teorie sull'improvvisazione,
riguardo queste citerà più volte deliberatamente in interviste, scritti,
e altre documentazioni, il suo approccio sistemico/cibernetico
in quelle che saranno le esperienze con il Gruppo. \\
Nel '60 si trasferisce a Roma Roland Kayn da vincitore del Prix de Rome,
dove dal '64 assieme ad Aldo Clementi e Franco Evangelisti
prende parte al Gruppo di improvvisazione Nuova Consonanza
del quale fece parte sino al '68,
ed è in quel periodo che Kayn ispirato dalle teorie della cibernetica iniziò a sperimentare
estensivamente con sistemi di autoregolazione basati su feedback loops,
non più solo come modelli formali per composizioni strumentali
ma anche come reti di generatori di segnale analogici. \\
% Europa - Francia
Sempre negli anni '50 in Europa, uno dei primi artisti nella storia dell'arte
ad evocare l'uso della cibernetica nei propri lavori è stato
Nicolas Schoeffer con il suo ciclo di lavori "spazio-dinamici", in acronimo
CYSP - Cybernetic Spatiodynamic.
In particolare Schoeffer ha creato la prima installazione ad implementare meccanismi
di auto-regolazione, il CYSP-1 \footcite{sanfilippovallefeedsys},
capace di essere sensibile all'ambiente esterno e a se stesso
grazie ad una serie di tecnologie offerte dalla compagnia Philips (fotocellule e microfoni),
questa prima scultura spazio-dinamica, è dotata di totale autonomia di movimento
(viaggio in tutte le direzioni a due velocità) e di rotazione assiale ed eccentrica
(messa in moto delle sue 16 lastre policrome pivotanti),
ed era capace di reagire sonoramente a questi stimoli riproducendo
una serie di registrazioni composte dal compositore francese Pierre Henry,
collaboratore di Pierre Schaeffer ed insieme a lui figura centrale nella nascita della Musique concrète.
Questa scultura è celebrata come una prima e prima opera di carattere cibernetico
che è entrata nel mondo dell'arte.

% America
cambiando continente e passando dall'Europa ad osservare cosa stava accadendo
in America in quegli anni, troveremo tanti altri rilevanti atti pioneristici,
come ad esempio quelli che sono stati i lavori di Louis e Bebe Barron.
Louis e Bebe Barron furono due compositori e pianisti che si
interessarono alla musica elettronica sin dal periodo della sua origine.
Intorno al '50 i due si trasferirono al Greewitch Village a New York
dove furono attivi in collettivi di musica sperimentale
collaborando con persone come John Cage ed altri.
I Barron trasformarono
la loro casa in una specie di studio di musica Elettronica
dove scrivevano soundtracks per film sperimentali.
Per Louis e Bebe il grande passo arrivò nel
'56, quando i due si ritrovarono a scrivere la soundtrack per il film:
\textit{Forbidden Planet},
questo sarà il primo film mainstream di Hollywood ad utilizzare una soundtrack
composta solamente ed interamente da elettronica.
L'elettronica di \textit{Forbidden Planet} è stata
costituita a partire da dei circuiti appositamente creati da Louis e Bebe,
i due deliberatamente ispirati dalle teorie cibernetiche di Wiener
dichiareranno: \footcite{dunbarlisteningcyb}

\begin{center}
\vspace{0.5cm}
\textit{What we did was pretty elementary: we would attach resistors and capacitors
to activate these
circuits... negative and positive feedback was involved - Wiener
talks about all that. The same
conditions that would produce breakdowns and malfunctions in machines,
made for some
wonderful music. The circuits would have a “nervous breakdown”
and afterwards they would be
very relaxed, and it all came through in the sounds they generated.}
\vspace{0.5cm}
\end{center}

I circuiti in retroazione erano destinati al corto circuito,
e utilizzati appositamente come materiale per la generazione acustica
di trame incise su nastro.
Altri importanti compositori di quel periodo sono al lavoro su composizioni
che sfruttano sistemi di Feedback, fra i più rilevanti:
John Cage, David Tudor, Robert Ashley, Gordon Mumma e Steve Reich. \footcite{sanfilippovallefeedsys} \\
Un secondo periodo costituito da un approccio sistemico più consapevole
che inizia a tracciare la strada per un pensiero ecosistemico della composizione,
in un certo senso si può affermare che inizi dal lavoro
di Alvin Lucier, che nel '69 scriverà quello che sarà un brano emblematico per
la cibernetica in musica \textit{I'm sitting in a room},
è un altro brano importante per quelle che sono
le logiche di interazione sistemiche fra uomo/macchina/ambiente che approfondiremo 
più avanti nel corso della tesi,
e che proprio come nel passaggio da cibernetica di primo 
ordine a quella di secondo ordine di cui parla Heinz Von Foerster, 
viene sancita l'interazione sistemica dove il musicista l'ambiente e lo strumento
divengono parti di un insieme complesso,
il sistema stesso di \textit{I'm sitting in a room} si osserva tramite l'ambiente 
nel corso della sua evoluzione, tracciando una storia di relazioni e interazioni 
con questo.
Nel brano di Lucier, un performer
recita in un microfono un testo che descrive il fenomeno che avverrà poco a poco,
la voce recitante nel microfono viene registrata e poi riprodotta da altoparlanti
posti nella stanza, il suono della registrazione riprodotta da questi altoparlanti
viene poi registrato nuovamente durante la riproduzione da un secondo supporto, e l'operazione
viene ripetuta in un in una casualità circolare di volta in volta
fino alla fine dove rimarranno
solo i contributi provenienti dalle frequenze di risonanza della stanza,
dalla voce e dalla catena elettroacustica,
dando vita nel loro insieme ad un processo molto lento di Feedback positivo dove
la natura non lineare del processo e degli agenti porterà di volta in volta ad un risutato
sempre differente.
Lucier racconta di essere stato originariamente ispirato
a creare I Am Sitting in a Room dopo aver partecipato a una conferenza al MIT,
in cui Amar Bose, imprenditore, ingegnere elettrico e tecnico del suono,
descrisse come testava le caratteristiche degli altoparlanti
che stava sviluppando, ri-diffondendo l'audio prodotto nella stanza e poi
riprendendolo tramite i microfoni, in quella che è la stessa casualità
circolare ripresa ed utilizzata artisticamente da Lucier. \footcite{lucierbose}
Dopo l'esperienza di Lucier, Nicolas Collins,
formatosi nella tradizione compositiva sperimentale con Alvin Lucier,
David Behrman e David Tudor, con i quali ha lavorato a stretto contatto,
nel '74 compone il brano \textit{Pea Soup} mentre è studente alla Wesleyan University.
In \textit{Pea Soup} una rete auto-stabilizzante di circuiti analogici
(originariamente tre Countryman Phase Shifter)
sposta il tono del feedback elettoacustico su una frequenza di risonanza
diversa ogni volta che questo si inizia a
costruire. Il suono familiare del fenomeno è sostituito da schemi instabili
che danno vita ad un raga site-specific
rispecchiando la personalità acustica della stanza. \footcite{collinspeasouphist}
Anche in questo lavoro appare chiaro come una certa sensibilità
nei confronti del fenomeno di Feedback si stia avvicinando ad una logica più vicina
a quella della cibernetica di secondo ordine.
Per citare un'ultima esperienza americana rilevante,
c'è infine il caso della Neural Synthesis di David Tudor.
David Tudor nel '89 incontrò il progettista e designer Forrest Warthman
dopo uno show a Berkeley,
che che lo introdusse all'idea di utilizzare reti neurali analogiche
per combinare tutta la sua complessa attività live-electronics in un unico computer.
\footcite{tudorneuralsynth}
Uno dei risultati è apprezzabile nei CD's "Neural Synthesis", 
di cui in uno sono presenti delle importanti note di copertina di Warthman:

\begin{center}
\vspace{0.5cm}
\textit{This recording combines the art of music, the engineering of electronics,
and the inspiration of biology. In it,
David Tudor orchestrates electronic sound in ways analogous to
our biological bodies' orchestration of consciousness...
The neural-network chip forms the heart of the synthesizer.
It consists of 64 non-linear amplifiers (the electronic neurons on the chip)
with 10240 programmable connections.
Any input signal can be connected to any neuron,
the output of which can be fed back to any input via on-chip or off-chip paths,
each with variable connection strength. The same floating-gate devices used in EEPROMs
(electrically erasable, programmable, read-only memories)
are used in an analog mode of operation to store the strengths of the connections.
The synthesizer adds R-C (resistance-capacitance) tank circuits on feedback
paths for 16 of the 64 neurons to control the frequencies of oscillation.
The R-C circuits produce relaxation oscillations.
Interconnecting many relaxation oscillators rapidly produces complex sounds.
Global gain and bias signals on the chip control the relative amplitudes of neuron oscillations.
Near the onset of oscillation the neurons are sensitive to inherent
thermal noise produced by random motions of electron groups moving through
the monolithic silicon lattice. This thermal noise adds unpredictability
to the synthesizer’s outputs, something David found especially appealing.
The synthesizer’s performance console controls the neural-network chip.
R-C circuits, external feedback paths and output channels.
The chip itself is not used to its full potential in this first synthesizer.
It generates sound and routes signals but the role of learner,
pattern-recognizer and responder is played by David, himself a vastly
more complex neural network than the chip}
\vspace{0.5cm}
Neural Synthesis No.6-9 liner notes by Forrest Warthman, Palo Alto 1995
\footcite{http://www.lovely.com/albumnotes/notes1602.html}
\vspace{0.5cm}
\end{center}

Ad oggi svariati compositori a partire dalle trame delineate dalle scienze complesse e
dai lavori citati, operano nell'ambito della musica elettronica con un approccio sistemico,
fra questi molti sono italiani. \\
Ci sono casi particolarmente rilevanti come quello di Agostino Di Scipio,
uno dei maggiori compositori con più contributi all'attivo,
da prima con i suoi studi sul caos e sui sistemi complessi in modo formale ad inizio
anni '90 \footcite{discipioiterated},
e poi verso metà anni '90 con l'inizio del lavoro sulla composizione ecosistemica
con il suo ciclo di lavori
- ecosistemico udibile, che come vedremo a seguito riprende
quelle che sono le tematiche della cibernetica di secondo ordine dove
il sistema si osserva tramite l'ambiente circostante.
O dell'(ex)allievo di Di Scipio, Dario Sanfilippo,
compositore e ricercatore con all'attivo recenti importanti pubblicazioni e lavori
nell'ambito dei sistemi autonomi DSP in musica, tematiche che rimandano
alle proposte che vanno dai Barron fino alla Neural Synthesis di Tudor,
seppure in forme più primordiali nel caso di questi ultimi,
così come Di Scipio nella sua sensibilità ecosistemica sembra rimandarci ai
lavori pionieristici di Lucier e Collins. \\
C'è poi il caso di Michelangelo Lupone, che a sua volta è stato maestro
di Agostino Di Scipio,
che con i suoi lavori di Feedback sulla materia è arrivato durante il corso degli
anni '90 allo sviluppo
pionieristico di strumenti aumentati in Feedback, quale ad esempio il Feed-Drum,
innovativo strumento elettroacustico a percussione.
E ci sono poi anche altri compositori internazionalmente riconosciuti
con all'attivo composizioni e ricerche rilevanti in lavori con il Feedback
e i sistemi autonomi, come: Andrea Valle, Giuseppe Silvi, Simone Pappalardo. \\
Parlando delle questioni italiane e tornando all'origine delle questioni romane,
nonostante la natura frammentaria e sottile della musica elettronica romana,
in effetti è comunque possibile individuare e tracciare una sorta di collegamento
che ci porta sin dalle prime suggestioni sulla cibernetica avute da Evangelisti
con gli altri membri di Nuova Consonanza, fino ad oggi.
Michelangelo Lupone ad esempio che abbiamo citato per i suoi lavori
e per esser stato maestro di Agostino Di Scipio, studia dal ’70 al ’79,
sotto la guida di Domenico Guaccero per la Composizione
e Giorgio Nottoli per la Musica elettronica.
Domenico Guaccero è a sua volta fra i fondatori, insieme ad Evangelisti
ed altri compositori
quali Aldo Clementi, Daniele Paris, Francesco Pennisi,
dell'Associazione di Nuova Consonanza.
Walter Branchi, noto anche lui
per aver preso parte al Gruppo di Improvvisazione Nuova Consonanza,
durante gli anni '80 darà vita a degli incontri internazionali su
Musica complessità, che radunavano compositori e scienziati di
tutto il mondo.
Di Scipio e Lupone in questo scenario, si sono interessati alle questioni
sul Feedback in contemporanea
portando avanti il discorso in maniera indipendente intorno alla fine degli anni '80,
Di Scipio nel 1989, scrive gli appunti a base di semplici funzioni
iterate da cui nacque poi il suo brano 
\textit{Fractus}, per viola e supporto digitale.
Lupone dal 1988, fonda ed inizia con il Centro Ricerche Musicali il suo lavoro
con team multidisciplinari di ricerca,
con la collaborazione di persone come Lorenzo Seno direttore scientifico del CRM.
Tutto questo mette in luce come le problematiche relative alla cibernetica
e i sistemi complessi siano stato un argomento molto sentito e vivo
nella prassi della composizione elettroacustica in tutto il mondo.

\subsection{Il Feedback}
\label{sec:Il Feedback}
Il feedback (o retroazione) è un concetto cibernetico che sta ad indicare
la capacità di un sistema di autoregolarsi tenendo conto degli effetti scaturiti
dalla modificazione delle caratteristiche del sistema stesso.
In termini appartenenti alla fisica, è la capacità di un sistema dinamico
di tenere conto dei risultati prodotti dal sistema per
modificare le caratteristiche del sistema stesso.
Quello che osserviamo in un ciclo di feedback è
il passato che viene influenzato dal presente,
e che sta per essere compensato dall'immediato
futuro. \footcite{sanfilippocasaesthetics}
Negli esseri viventi, ad esempio, i sistemi a retroazione negativa e positiva
sono ampiamente utilizzati per regolare l'omeostasi dell'organismo.
Esistono idealmente due tipologie di Feedback:
\\ \\
- a Retroazione Positiva
\\
- a Retroazione Negativa
\\ \\
In un sistema soggetto a questi stimoli,
il feedback positivo può creare una risposta con crescite o decadimenti esponenziali
dal punto di equilibrio naturale del sistema.
Questo viene spesso visto come una reazione a catena
dove le perturbazioni che tendono verso una certa direzione
saranno amplificate ricorsivamente dal
sistema stesso, determinando ulteriori spostamenti nella medesima direzione.
D'altra parte invece, i circuiti di feedback negativo hanno
un comportamento complementare al feedback positivo e tendono ad oscillare
attorno ad un punto di stabilità.
In effetti, i sistemi di feedback negativo sono in equilibrio dinamico
e svolgono un azione di controbilanciamento nei confronti degli stimoli esterni
per mantenere uno stato di equilibrio. \footcite{sanfilippocasaesthetics}
Nel corso della tesi ritorneremo su questi temi
riguardo le proprietà del feedback, affrontandoli di volta in volta
nei casi specifici.
Per ora ci basti pensare che nel controllo di un sistema complesso
come può essere quello del feedback elettroacustico,
introdurre delle linearità tramite retroazione all'interno del ciclo di
feedback potrebbe voler dire costringere la complessità
a dei comportamenti prevedibili, uno stato di convergenza verso l'equilibrio. \\
Un esempio pratico e conosciuto è quello dell'intonazione del fenomeno Larsen,
che da un comportamento complesso della sorgente e del ricettore arrivando 
alla saturazione per autoscillazione giunge ad uno stato stazionario, di stabilità.
Mentre introdurre delle non linearità nel sistema tramite la retroazione,
potrebbe voler dire portare il sistema verso comportamenti non più prevedibili,
in divergenza dall'equilibrio dello stato stazionario,
e in alcuni casi verso la soglia del caos.
Questi due tipi di comportamento possono essere ottenuti per l'appunto
sia velocizzando che rallentando questi processi
in maniera dipendente dal caso specifico.
Un secondo esempio esemplare è quello dei filtri digitali nell'audio,
questi possono essere pensati come un valido strumento
di contrasto o favoreggiamento rispetto a questo tipo di comportamenti,
dove se si allineano le fasi si creano dei poli,
mentre se si disallineano si punta generalmente alla complessità del sistema.
Di fatto la storia delle tecnologie elettroacustiche ha più in generale
da sempre incorporato il principio
del feedback sin dalle sue origini, basti pensare a tecnologie
come il triodo, chiamato inizialmente valvola audion di Lee De Forest,
o i circuiti di feedback negativo - negative feedback amplifier - di Harold Black
\footcite{echodiscipio}.
