\section{Primo appendice}

In questo appendice sono riportati tutti i codici dei sistemi
e le relative librerie di oggetti trattati nel corso della tesi. \\
\verb| Audible_Ecosystemics_2.dsp | è il file principale di una implementazione 
completa del sistema di Audible Ecosystemics 2 di Agostino Di Scipio, sviluppato con il supporto
e il prezioso aiuto di Dario Sanfilippo, nel Debug del sistema e sviluppo della libreria del codice; 
e conseguente al prezioso lavoro svolto con i miei compagni nella Classe di esecuzione ed interpretazione 
della musica elettroacustica di Giuseppe Silvi, che mi ha permesso di approfondire le tematiche
discusse nel corso della tesi riguardo al brano e poter sviluppare strategie efficaci per questa implementazione.
\verb| aelibrary.lib | è invece la libreria degli oggetti utilizzati nella implementazione del sistema
di Audible Ecosystemics 2 di Agostino Di Scipio (oggetti discussi anche nella tesi). \\
\verb| RITI_v1_CelloC2.dsp | è il file principale della implementazione 
del mio sistema di RITI v.1.0. \verb| RITI.lib | è invece la libreria degli oggetti utilizzati nella 
implementazione del mio sistema di RITI v.1.0. \verb| SpectreLists.lib | è infine una libreria
con i risultati delle analisi condotte sulle note del violoncello, utilizzata per passare le informazioni
ai filtri passabanda nel sistema di RITI v.1.0. \\
\clearpage


\begin{center} \Large \verb| Audible_Ecosystemics_2.dsp | \normalsize \\
    \vspace{0.2cm} \end{center}
\lstinputlisting[breaklines, frame=trBL]{codes/Audible_Ecosystemics_2.dsp}
\clearpage

\begin{center} \Large \verb| aelibrary.lib | \normalsize \\
    \vspace{0.2cm} \end{center}
\lstinputlisting[breaklines, frame=trBL]{codes/aelibrary.lib}
\clearpage

\begin{center} \Large \verb| RITI_v1_CelloC2.dsp | \normalsize \\
    \vspace{0.2cm} \end{center}
\lstinputlisting[breaklines, frame=trBL]{codes/RITI_v1_CelloC2.dsp}
\clearpage

\begin{center} \Large \verb| RITI.lib | \normalsize \\
    \vspace{0.2cm} \end{center}
\lstinputlisting[breaklines, frame=trBL]{codes/RITI.lib}
\clearpage

\begin{center} \Large \verb| SpectreLists.lib | \normalsize \\
    \vspace{0.2cm} \end{center}
\lstinputlisting[breaklines, frame=trBL]{codes/SpectreLists.lib}
\clearpage

\section{Secondo appendice}

In questo appendice sono riportati i codici utilizzati per le analisi 
spettrali del violoncello effettuati in Python per il sistema RITI v.1.0. \\
\verb| FFT_range.py | il primo prototipo di questo codice è stato sviluppato
in collaborazione con il mio collega e amico Edoardo Staffa. 
L'analisi DFT del file audio può essere effettuata scegliendo il numero di Bins e 
il Range in frequenza da analizzare (il Bandwidth è ricavato sulla base di questi dati),
l'output dell'analisi è sia un plot di uno spettrogramma in formato .html navigabile,
che un file di testo con tutti i dati: frequenze, ampiezze e Bandwidth.
\verb| sort_amplitudes.py | prende la lista generata ed effettua un sort
decrescente delle liste a partire dalle frequenze con i picchi di ampiezza più 
elevati a quelli meno elevati. \verb| filter_frequencies.py | elimina le frequenze 
dalle liste troppo vicine ai picchi più elevati (impostato di default a 40Hz).
\verb| sort_frequencies.py | effettua il sort per frequenze in ordine crescente.
\verb| makelib.py | infine converte le liste .txt processate in un file .lib leggibile da Faust.
\clearpage


\begin{center} \Large \verb| FFT_range.py | \normalsize \\
    \vspace{0.2cm} \end{center}
\lstinputlisting[breaklines, frame=trBL]{codes/FFT_range.py}
\clearpage

\begin{center} \Large \verb| sort_amplitudes.py | \normalsize \\
    \vspace{0.2cm} \end{center}
\lstinputlisting[breaklines, frame=trBL]{codes/sort_amplitudes.py}
\clearpage

\begin{center} \Large \verb| filter_frequencies.py | \normalsize \\
    \vspace{0.2cm} \end{center}
\lstinputlisting[breaklines, frame=trBL]{codes/filter_frequencies.py}
\clearpage

\begin{center} \Large \verb| sort_frequencies.py | \normalsize \\
    \vspace{0.2cm} \end{center}
\lstinputlisting[breaklines, frame=trBL]{codes/sort_frequencies.py}
\clearpage

\begin{center} \Large \verb| makelib.py | \normalsize \\
    \vspace{0.2cm} \end{center}
\lstinputlisting[breaklines, frame=trBL]{codes/makelib.py}
\clearpage
