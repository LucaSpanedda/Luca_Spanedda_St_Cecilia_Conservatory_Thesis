%% TITOLO
\section{RITI: un sistema caotico ibrido}
\label{sec:RITI: un sistema caotico ibrido}

RITI : Room Is The Interface è un Brano basato su un CAS autonomo \textit{site-specific},
dove la personalità acustica della stanza dove viene eseguito il brano
viene riflessa in variazioni del comportamento del sistema stesso.
L'idea dell'acronimo RITI nasce come omaggio a: Sound is the interface, 
il famoso paper di Agostino Di Scipio, che introdusse per la prima volta
una prospettiva sistemica sull'auto-organizzazione e sulla capacità
di un sistema di auto-osservarsi tramite l'ambiente circostante. \\





%% TESTO
- parlare del brano e della sua realizzazione +
Codici Faust degli agenti e delle varie parti.
Le analisi condotte e il contributo proveniente dai lavori
di Agostino Di Scipio e Dario Sanfilippo.
In coda al capitolo rimandare al codice completo in Appendice A e plot
grafici di Faust in Appendice B -
+ Partitura dopo gli appendici (o appendice C)

\begin{center} \vspace{0.5cm} \Huge - Da Continuare - \normalsize \vspace{0.5cm} \end{center}