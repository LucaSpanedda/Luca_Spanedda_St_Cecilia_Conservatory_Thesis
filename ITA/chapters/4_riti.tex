%% TITOLO
\section{RITI: un sistema caotico ibrido}
\label{sec:RITI: un sistema caotico ibrido}

RITI : Room Is The Interface è un Brano basato su un CAS autonomo \textit{site-specific},
dove la personalità acustica della stanza dove viene eseguito il brano
viene riflessa in variazioni del comportamento del sistema stesso.
L'idea dell'acronimo RITI nasce come omaggio a: Sound is the interface, 
il famoso paper di Agostino Di Scipio, che introdusse per la prima volta
una prospettiva sistemica sull'auto-organizzazione e sulla capacità
di un sistema di auto-osservarsi tramite l'ambiente circostante. \\

\subsection{Motivazioni}
\label{sec:Motivazioni}

In questi ultimi anni di studi musicali, sono emerse nel mio fare musica 
delle questioni per me di fondamentale importanza che hanno accompagnato il mio operato
e che giustificano la mia esigenza di essermi interessato ai CASes in musica. \\
In primo luogo indagare rispetto al rapporto che esiste fra l'interprete, la notazione, e lo strumento musicale;
che può essere vista come una parte contenuta all'interno della relazione fra uomo-macchina-ambiente di cui parla
Agostino Di Scipio. \\
Le prime forme di semiografia musicale possono essere rintracciate in diverse civiltà 
che risalgono addirittura attorno al 2000 A.C.,
sin d'allora l'uomo ha posto al centro della sua indagine l'interrogativo sul come fissare sempre meglio 
successioni strutturate di suoni semplici o complessi tramite una serie di parametri nel dominio discreto (la notazione), 
e che interpretati in un secondo momento potessero restituire l'idea iniziale in una resa 
acustica nel dominio del continuum temporale.
Questi parametri principalmente impiegati nella semiografia musicale Occidentale, 
sono stati fino all'incirca ad inizio del XX Secolo: 
le altezze dei suoni, le intensità, i ritmi, le durate, ed infine il timbro desiderato, 
quest'ultimo descritto richiedendo l'utilizzo di un certo strumento musicale o di specifici gruppi strumentali.
Tuttavia il significato stesso del termine musica non è comunque univoco ed ha avuto (e ha ancora oggi) 
diverse accezioni utilizzate nei vari periodi storici.
Difatti con l'arrivo del XXI Secolo, sono state messe in crisi le nozioni di strumento musicale e di semiografia musicale, 
in quanto la pratica del comporre non si è più solamente limitata al comporre 
con i timbri dei diversi strumenti musicali, ma si è aperta anche alla possibilità del comporre stesso nel timbro, 
spingendo i compositori a cercare vie alternative ed alle volte individuali.
Riguardo gli strumenti musicali invece, se la voce è difatti il primo strumento dell'uomo in quanto 
abita il corpo stesso che ne è il custode, la macchina che nel suo senso generico è qualsiasi strumento 
il cui moto relativo trasmetta o anche amplifichi la forza umana, 
sin dall'antichità si è prestata all'idea di strumento musicale come a qualcosa per assistere l'uomo nella 
sua intenzione del fare musica secondo l'organizzazione dei parametri esposti precedentemente; 
con la composizione nel timbro viene invece a mancare l'idea 
di una musica composta solamente da queste tipiche relazioni tramite gli strumenti musicali, 
rivalutando l'idea stessa che si ha di uno strumento musicale, aperta ora di fatto alla possibilità 
del poter comporre quanto il timbro stesso, anche lo strumento che lo produce. \\
Come dice Trevor Wishart nel suo libro On Sonic Art:

\begin{center}
    \vspace{0.5cm}
    \textit{In this book i will suggest that the logic of this assertion is inverted.
    It is notability which determines the importances of pitch, rythm, and duration and
    not vice versa and that much can be learned by looking at 
    musical cultures without a system of notation... 
    In this book, I will suggest that we do not need to deal with a finite set of
    possibilities. The idea that music has to be built upon  finite lattice and the
    related idea that permutational procedures are a valid way to proceed will
    be criticised here and a musical methodology developed for dealing with a
    continuum using the concept of transformation}\footcite{wishart_on_sonic_art}
\vspace{0.5cm}
\end{center}

Come comportarsi quindi davanti ad una condizione in cui viene messa in crisi l'idea stessa
che si ha di musica, di notazione musicale, e di strumento musicale?
La mia indagine mi ha portato a rivolgere l'attenzione al rapporto con la notazione 
e gli strumenti musicali in quelle opere dei compositori contemporanei, 
che raccogliendo le ceneri di questa crisi del XXI Secolo, 
hanno deciso di approfondire l'idea dietro l'utilizzo
di uno strumento musicale con cui fare musica, non più come 
qualcosa di funzionale al riprodurre dei parametri annotati in partitura a priori,
ma come un sistema che offre una serie di possibilità; 
in cui lo strumento diviene il corpo acustico su cui indagare 
tutta una serie di relazioni e gesti possibili dell'interprete, 
alla ricerca di un'esplorazione sonora dello spazio del sistema.
In tal senso Trevor Wishart suggerisce di guardare alla Teoria delle Catastrofi 
di René Thom, assumendo che le regole formali di una musica possono
essere ricavate dall'osservazione di uno strumento stesso come se questo fosse un sistema.
Ho rintracciato questo tipo di atteggiamento nei confronti dello strumento
e della notazione in opere come: Pression di Helmut Lachenmann,
dove la partitura viene ad indicare come agire sul corpo dello strumento
schiudendone nella sua totalità il suo mondo sonoro accessibile,
o nell'opera Necessità d'interrogare il cielo di Giorgio Netti,
dove a partire dagli elementi della notazione tradizionale, lo studio condotto 
sul corpo dello strumento e degli esiti acustici del gesto, scendono così tanto 
"ad un basso livello di articolazione" da schiuderne in questo tutta una serie di nuove possibilità
acustiche, o anche nei lavori stessi di Wishart, come nel ciclo Vox 
dove la voce viene esplorata in tutte le sue ambiguità rendendo
complesso il collegamento fra il suono e la sorgente che lo produce. \\
Se è vero quindi come dice Wishart che le regole formali di una musica possono
essere ricavate dall'osservazione di un sistema, 
la pratica del disegnare un sistema autonomo (la macchina) con cui fare musica,
ne riflette di conseguenza la quantità di informazione che questo contiene al suo interno,
e di conseguenza il tipo di relazioni e interazioni con il suo ambiente che 
ci interessano dal punto di vista compositivo.\\
In secondo luogo indagare rispetto al rapporto conflittuale che l'uomo ha con la tecnologia
nella società contemporanea. 
In effetti, l'uomo ha da sempre un rapporto conflittuale con la tecnologia,
se sin dai tempi di Platone emergevano queste problematiche e preoccupazioni
relative alla discretizzazione del pensiero con la scrittura
come possiamo leggere in Fedro, 
giungendo alle paure più contemporanee di Norbert Weiner riguardo la tecnologia
esposte nel sua Introduzione alla Cibernetica, l'uso umano degli esseri umani,
l'idea di una società contemporanea composta da interazioni sistemiche fra l'uomo e la macchina
in cui il primo è succube della seconda, sembra in qualche modo essersi rivelata profetica e realizzata.
In risposta a questa condizione, disegnare i propri sistemi e di conseguenza riappropriarsi della tecnologia,
rappresenta dalla mia prospettiva personale un modo costruttivista di rispondere ai problemi del mondo contemporaneo
attraverso la pratica musicale.

\begin{center}
    \vspace{0.5cm}
    \textit{The increasingly frenetic accumulation of sounds and
images is a part of this intensification, aiming to fill the void of
a present that is forever sinking into the past. At every
moment, the world must be supplied with novel sensory
material generated to replace existing material that has
already become obsolete. The exaltation of the present as the
promise of pleasure Is therefore never really meant to be actualised,
since it will immediately be obsolesced by a continuously
renewed influx of promises.
Needs, cravings, and desires
are endlessly rebooted.
The present moment, as constructed by our society, is
nothing but a procedure of forgetting, a catalyst for amnesia,
and as such it enables the repetition which, in turn, through
the presentation of novelty, allows us to distance ourselves
3 little further from the past. And this writing and rewriting
of the present—as if it were a palimpsest—continues to
accelerate, demanding that we forget at an ever higher fre-
quency. What was is of no concern; all has beens are eliminated
in favour of what is, with little regard for what will be.
Where the modern era anticipated a present yet to come,
transposing current momentum into future accomplishments,
our postindustrial, postmodern era sees the future as
nothing but a vague, indeterminate site hosting an indistinct
cloud of promises and signs as desirable as they are deadly.}\footcite{francois_jbonnet_after_death}
\vspace{0.5cm}
\end{center}

Per concludere le mie motivazioni, l'ultimo aspetto che ho reputato importante affrontare è la forma,
nei suoi aspetti Macro-temporali e Micro-temporali. \\
Se è vero come sostiene François J. Bonnet, ed altri filosofi come Mark Fisher e Slavoj Žižek,
che la nostra società contemporanea ha in qualche modo normalizzato
una velocità schizofrenica nelle cose, dove tutto deve essere estremamente funzionale per motivi pratici,
la riappropriazione del tempo nella forma, dell'ascolto, ed infine dello spazio, diviene per me un elemento fondamentale
nella composizione,
dove tutti gli elementi del sistema hanno un ruolo condiviso che contribuisce a formare la complessità nel suo insieme.
A tal proposito ho sempre trovato interessante come un piccolo cambiamento all'interno di un sistema 
possa portare a cambiamenti radicali
nella Macroforma, proprio come nella sensibilità alle condizioni della teoria del caos, 
che ci illustra come in un sistema caotico, a variazioni infinitesime delle condizioni iniziali 
corrispondono variazioni significative del comportamento futuro. 
In tal senso è diventato per me un aspetto importante lavorare con
sistemi emergenti che possano rendere udibili questi processi di continua trasformazione
del suono. \\
E a tal proposito Stockhausen, nella sua teoria della "Forma unificata del tempo", 
spiega come in tal senso la differenza fra i Micro-elementi nella forma, 
e la Macroforma, non consista tanto in diversi ruoli e compiti del comporre,
quanto a diversi livelli di percezione del tempo, 
poiché altezza e ritmo sono fondamentalmente la stessa cosa percepita a diverse scale temporali,
come illustra nella sua composizione Kontakte realizzata fra il 1958 ed il 1960.

%% TESTO
%% - parlare del brano e della sua realizzazione +
%% Codici Faust degli agenti e delle varie parti.
%% Le analisi condotte e il contributo proveniente dai lavori
%% di Agostino Di Scipio e Dario Sanfilippo.
%% In coda al capitolo rimandare al codice completo in Appendice A e plot
%% grafici di Faust in Appendice B -
%% + Partitura dopo gli appendici (o appendice C)