%% TITOLO
\section{Ecosistemi Udibili}
\label{sec:Ecosistemi Udibili}

Uno degli esiti musicali più importanti che ha portato con sé 
il paradigma della complessità, riguarda essenzialmente l'aver smesso di 
scrivere musica per strumenti interattivi con cui farla (e viceversa), 
ed esser passati invece al  comporre le interazioni tramite gli strumenti. \\
Fra le altre cose, nell'insieme delle parole: Sistema Tonale, 
che sta a significare quell'insieme delle regole di organizzazione ed uso dei
suoni maggiormente utilizzati per produrre musica nella tradizione 
musicale Occidentale, è frequente che passi in secondo piano la parola 
sistema, che assume invece un significato centrale nel contesto di questa tesi;
difatti molti compositori non sono passati solamente al comporre le interazioni, 
ma anche al comporre i sistemi stessi con cui fare la propria musica.
Come accennato nell'abstract, le interazioni che ho ritenuto importante affrontare 
all'interno dei sistemi dei compositori cibernetici sono di due tipi:
di sistemi che interagiscono con l’ambiente circostante appartenente al mondo fisico, 
e di sistemi che interagiscono con il proprio ambiente nel mondo digitale, 
ma prima di entrare nel merito del primo caso qui,
è doveroso spendere qualche parola su cosa
si intenda per ambiente. \\
Nessun sistema è separabile e isolabile dall'ambiente circostante, 
a prescindere dal fatto che il suo spazio vitale, sia nel mondo fisico o digitale. 
Proprio come ha mostrato Heinz Von Foerster non si può parlare di auto-organizzazione
se non ci si riferisce essenzialmente a un ambiente che racchiuda il sistema al suo interno.
E nella pratica musicale, la sensibilità nel comporre le interazioni
è spesso lontana dall'idea del voler interagire con una consapevolezza effettiva del 
puntare ai cambiamenti di stato all'interno di un sistema con cui si interagisce,
e a maggior ragione viene trascurata la possibilità di interagire con sistemi che abbiano
la capacità di auto-osservarsi non dipendendo più direttamente dal controllo dell'esecutore 
secondo delle modalità prettamente lineari, 
ma dall'ambiente, 
con la capacità di guardare al proprio stato interno per creare autonomamente le interazioni.
In particolare, nei sistemi auto-osservanti quello che si manifesta quando un sistema entra in interazioni non distruttive
con il proprio ambiente circostante, che coincide nella sostanza anche nel suo spazio vitale, è un Ecosistema. \\
In tal senso, i lavori del ciclo Ecosistemico Udibile di Agostino di Scipio, 
trovano fondamento a partire da fenomeni e relazioni che possono esistere e manifestarsi solamente nell'ambiente 
circostante da cui prende vita il sistema, 
che nel suo caso diventa proprio lo spazio acustico reale. 
Lavori come lo studio sul feedback, lo studio sul rumore di fondo, lo studio sulle risposte all'impulso, o
lo studio sul silenzio, sono nella loro essenza dei sistemi che hanno come seme 
della loro morfogenesi un determinato 
comportamento appartenente allo spazio acustico reale, delimitato in questi studi da una stanza
che ne racchiude al suo interno tutti gli agenti e le relazioni possibili,
per costruirne una storia di relazioni dove il sistema si osserva attraverso lo spazio fisico, 
(ambiente) e si manifesta e vive solo attraverso di esso.
In questo senso l’interprete, lo spazio, e gli ascoltatori, 
divengono essi stessi parte integrante del sistema, dove l’ascolto, diviene anche esso
parte dell’insieme delle cose e delle relazioni che lo costituisce. \\

\subsection{L'interazione Uomo-Macchina-Ambiente}
\label{sec:L'interazione Uomo-Macchina-Ambiente}

Qual'è dunque l'esigenza alla base del comporre le interazioni invece 
che limitarsi al comporre musica per strumenti interattivi? 
Superare la classica relazione uomo-macchina dominante nella tradizione musicale, 
iniziando a pensare invece allo sconfinato universo delle nuove possibilità della complessità.
Secondo Agostino Di Scipio, questo avviene in primo luogo grazie 
alla possibilità che la macchina possa rappresentarsi
senza mediazione umana, e attraverso l'ambiente circostante, \footcite{discipio_polverisonore_2016}
e dunque poi alla possibilità di stabilire un flusso di relazioni macchina-ambiente. 
Consentendo in secondo luogo al musicista e performer 
la possibilità di potersi aprire ad un flusso di relazioni 
complesso fra uomo-macchina-ambiente dove le tre sono fortamente connesse e interdipendenti 
l'una dall'altra.

\begin{center}
\vspace{0.5cm}
\includegraphics[width=8cm]{figures/uomo_macchina_ambiente.png} \\
{Interazione Uomo-Macchina-Ambiente} \\ 
\vspace{0.5cm}
\end{center}

\begin{center}
\vspace{0.5cm}
\textit{la mossa decisiva è: passare da un lavoro che mira a usare mezzi interattivi per creare forme sonore desiderate ad
un lavoro che mira a creare le interazioni desiderate e ad ascoltarne le tracce udibili. Nel secondo caso, si tratta
di progettare, implementare, e rendere operativo un reticolo di componenti interconnesse il cui comportamento
sonoro emergente si può chiamare musica.} \footcite{discipio_polverisonore_2016}
\vspace{0.5cm}
\end{center}

La modalità con cui andrò a discutere in questo capitolo la composizione di interazioni ecosistemiche, 
è ponendo il focus su un lavoro di Agostino Di Scipio, l’Ecosistemico Udibile n.2, studio sul Feedback. 
Partendo dalla partitura e dai suoi scritti, ed implementando e analizzando
il ruolo ed il compito dei singoli agenti all'interno del sistema 
che compongono nella loro totalità l'intero ecosistema del brano. 

\subsection{Il Meccanismo LAR}
\label{sec:Il Meccanismo LAR}
Il feedback elettroacustico (effetto Larsen), che abbiamo già detto
esser stata una delle risorse centrali dei primi compositori cibernetici,
è la condizione di partenza su cui Agostino Di Scipio opera per la costruzione degli Ecosistemi Udibili.
Volendoci addentrare ora verso una spiegazione più tecnica del feedback elettroacustico,
possiamo citare una definizione che Di Scipio ha esposto in un suo articolo
pubblicato presso la rivista Online: ECHO dell’ Orpheus Institute, in
Ghent. \footcite{di_scipio_relational_2022}

\begin{center}
\vspace{0.5cm}
\textit{A condenser microphone (M1) and a dynamic loudspeaker (L1) stand in the performance place
(S), few or several meters apart, maybe not too far from walls (or curtains, or other larger
surfaces). They are connected (through one or more amplification stages) to realize a very
basic electroacoustic chain: M1→L1→S. There’s no sound M1 should capture, though, no sound
source save the minimal, barely audible turbulence of the background noise, in a situation of
‘silence’. This ‘sound-of-nothing’ is amplified and heard through L1, whence it comes back in
S. \\
If amplification suffices, the L1 sound feeds back into M1 and the chain design closes onto
itself, making a ‘reinjection’ circuit – a feedback loop. The amplitude level, the
transductive technical features of M1 and L1, their relative distance, the distance from
walls, etc. – all of that (and much more) sets the actual feedback loop gain. With
not-too-high gain levels, what is engendered is an audible nuisance, a kind of ‘halo’: the
sound reinjection decays more or less rapidly, in a kind of badly sounding, spectrally uneven
reverb effect. With higher gain levels, the loop eventually enters a self-oscillatory regime,
it may ‘ring’ or ‘howl’, as is often said. Because of the iterated reinjection, the barely
audible but spectrally wide background noise accumulates in the loop and finally (quickly)
yields an increasingly louder sustained sound of narrower spectrum – this is often heard as a
peaking tone of definite pitch, or a tone cluster. That’s the Larsen effect: a self-sustaining
feedback resonance occasioned by a positive feedback loop (FB+) (‘positive’ here means greater
than unit gain).}
\vspace{0.5cm}
\vspace{0.5cm}
\end{center}

\begin{center}
\vspace{0.5cm}
\includegraphics[width=8cm]{figures/larsen_feedback_scheme.png} \\
{Schema M1→L1→S} \\ 
\vspace{0.5cm}
\end{center}

L'effetto Larsen: (dal nome del fisico Søren Absalon Larsen
che per primo ne scoprì il principio), detto anche feedback elettroacustico,
come abbiamo appena letto è un fenomeno di retroiniezione che tende idealmente 
ad un'accumulazione infinita, che viene poi limitata in realtà dalla saturazione dei sistemi 
che la generano (relativi alla potenza massima, all'amplificazione, nonché alla sensibilità dei
trasduttori e all'elasticità delle membrane). Che può anche essere oltre al microfono, un
pick-up di uno strumento musicale elettrico, come una chitarra o un basso, o un trasduttore di
altra natura... 
Esistono anche principi di Feedback acustico oltre che elettroacustico, 
come la risonanza acustica o la risonanza per simpatia,
che sono i principi su cui si basa il funzionamento di quasi tutti gli strumenti musicali,
ma doverose annotazioni a parte, non ne entrerò in questa sede nel merito del Feedback acustico
ma di quello elettroacustico.
Tornando dunque all'articolo di Agostino Di Scipio:

\begin{center}
\vspace{0.5cm}
\textit{In common sound engineering practice, audible feedback phenomena are a nuisance, a problem one
should get rid of or substantially minimize. When direct level manipulation is not enough, one
resorts to hard-limiting circuits, ‘feedback killers’ and alike devices... 
In a different attitude, one may instead consider feedback as
a resource, a deliberately designed sound-making mechanism one can play with.} \footcite{di_scipio_relational_2022}
\vspace{0.5cm}
\end{center}

per utilizzare il feedback come una risorsa,
appare chiara la necessità di un intervento riguardo la sua crescita che porta alla saturazione dei sistemi, 
in tal senso una delle possibili soluzioni è quella di poter far calcolare
al computer in tempo reale tramite diverse tecniche di \textit{amplitude following} la stima dell'ampiezza
del segnale in ingresso, ed utilizzare conseguentemente la \textit{feature extraction} come segnale di controllo in retroazione negativa al sistema di Feedback elettroacustico.
Questo meccanismo di controbilanciamento del guadagno
del fenomeno è chiamato da Agostino Di Scipio col nome di LAR: \textit{Audio Feedback with Self-regulated Gain},
e può essere implementato in DSP in diverse modalità e configurazioni,
ognuna con le sue diverse caratteristiche ed esiti. \\

\begin{center}
\vspace{0.25cm}
\includegraphics[width=4cm]{figures/controlled_larsen_feedback.png} \\
{Schema del meccanismo LAR} \\ 
\vspace{0.5cm}
\end{center}

Nella tradizione della computer music, ci sono in effetti
diversi modi per realizzare un algoritmo di controbilanciamento 
del feedback in tempo reale.
Alcuni di questi possono riguardare controbilanciamenti nel dominio della frequenza,
con tecniche di filtraggio automatizzate (adattive) che
hanno il compito di eliminare dallo spettro la presenza dell'autoscillazione prodotta dal Larsen \textit{Larsen Suppressors},
o come nel nostro caso d’interesse possono riguardare controbilanciamenti in ampiezza
automatizzati, che non permettano al Feedback di avere un guadagno troppo troppo elevato 
e giungere conseguentemente ad uno stato di saturazione del sistema.
Algoritmi di \textit{amplitude-following} possono essere implementati 
con la media \textit{RMS}, con finestre variabili o fisse, 
filtri \textit{Peakholder} che mantengano il valore di picco, o di altra natura. \\
Tratterò ora  una parte più operativa, discutendo l'implementazione
di alcune di queste tecniche nel linguaggio di programmazione FAUST (Grame), 
che è l'ambiente in cui ho scritto i codici di tutti i lavori trattati
in questa tesi e le relative compilazioni, diagrammi e softwares. \\
\footnote{FAUST (Grame) (Functional Audio Stream), 
è un linguaggio di programmazione specifico per il Digital Signal
Processing sviluppato da Yann Orlarey, Dominique Fober, e Stephane Letz nel
2002. Nello specifico, FAUST è un linguaggio di programmazione ad alto livello
scritto in C++, che permette di tradurre delle istruzioni date e create 
appositamente per il digital signal processing (DSP), in un largo raggio di linguaggi
di programmazione non specifici per il dominio dell’Audio Digitale.} \footcite{https://faust.grame.fr/} \\
Il modo più semplice per mantenere l’effetto Larsen in uno stato stazionario, è attraverso un
valore costante ricavato dalla crescita del segnale in ingresso 
che controbilanci l’ampiezza del circuito di Feedback, come in figura del meccanismo LAR;
la stima di questa costante avviene tramite un algoritmo di analisi.
Il modo più semplice per implementare un algoritmo di analisi di questo tipo
è attraverso un \textit{Peakholder} che nella sua forma basilare consiste in una finestra 
di campioni idealmente infinita IIR \textit{Infinite Impulse Response},
che ha il compito di confrontare il valore assoluto del campione in ingresso con il suo precedente,
il maggiore fra i due nella comparazione viene mandato sia in uscita che in ingresso 
in retroiniezione alla funzione stessa di comparazione, 
in questo modo si sfrutta un principio di feedback per l'accumulazione
del valore massimo, in un'analisi campione per campione.

\vspace{0.5cm}
\begin{lstlisting}
// import faust standard library
import("stdfaust.lib");

// Peak Max with IIR filter and max comparison
peakmax = loop
    with{
        loop(x) = \(y).((y , abs(x)) : max) ~ _ ;
    };
    
process = _ : peakmax;
\end{lstlisting}

\begin{center}
    \includegraphics[width=12cm]{figures/PeakholderIIR.pdf} \\
    {Algoritmo e Schema del \textit{Peakholder} ad 1 campione di ritardo} \\ 
    \end{center}
\clearpage 

\begin{lstlisting}
// import faust standard library
import("stdfaust.lib");
        
// LAR with Peak Max - IIR filter and max comparison
peakmax = loop
with{
    loop(x) = \(y).((y , abs(x)) : max)~_;
};
        
LARpeakmax = _ <: (_ * (1 - (_ : peakmax)));
process = _ : LARpeakmax;
\end{lstlisting}

\begin{center}
    \includegraphics[width=12cm]{figures/LARpeakmax.pdf} \\
    {Algoritmo e Schema del LAR con \textit{Peakholder} ad 1 campione di ritardo} \\ 
\end{center}
\vspace{0.5cm}

Questo tipo di algoritmo presenta comunque alcuni problemi: non avendo
una funzione di smoothing del segnale, si verificano problemi di segnali di differenza a banda
molto larga che possono generare aliasing e contributi spuri che tendono a permanere nel
segnale complessivo.
Oltre a questo, variazioni del comportamento del Feedback sono dipendenti dalla grandezza
della finestra di osservazione, da eventuali cofficenti di feedback inseriti nella
retroazione del \textit{Peakholder} e da altri tipi di implementazioni di tecniche 
\textit{amplitude-following}. Per questi motivi si sceglie più frequentemente di utilizzare algoritmi
adattivi, proprio come nel contesto del meccanismo LAR implementato nel Feedback Study,
dove il comportamento adattivo dell'\textit{amplitude-follower} è cruciale
per la dinamica adattiva ed auto-regolatoria del sistema.\\
Il meccanismo LAR è essenzialmente nel cuore della live performance del Feedback Study,
senza di questo non sarebbe possibile creare una condizione favorevole per procedere alle 
conseguenti trasformazioni del suono, ed ad una condizione favorevole affinché il 
sistema possa osservarsi tramite l'ambiente circostante. 
In tal senso vale la pena procedere osservando da vicino come viene richiesto in partitura 
di implementare questo meccanismo.

\begin{center}
\includegraphics[width=14cm]{figures/LARfeedbackstudy2017.pdf} \\
{Estratto dalla partitura di \textit{Audible Ecosystemics n.2/ Feedback Study (2003)} \\
revisione del 2017 - Agostino Di Scipio} \\ 
\vspace{0.5cm}
\end{center}

Come scritto in partitura, è da notare in prima istanza
come il microfono serva sia da sorgente per l'alimentazione del Larsen che
da canale che riceve l'informazione necessaria per il meccanismo auto-regolatorio,
fin qui è tutto uguale rispetto allo Schema del meccanismo LAR.
La sostanziale differenza consiste invece nell'aggiunta di un secondo microfono,
quando viene introdotta questa condizione di disaccoppiamento si 
permette al meccanismo di essere più responsivo ai piccoli cambiamenti nell'ambiente
e di rispondere diversamente agli stessi stimoli da diversi punti di osservazione del sistema.
Nel SIGNAL FLOW 1b, due microfoni in ingresso (mic1 e mic2) 
vengono limitati in banda da un filtro Highpass ed 
un filtro Lowpass, rispettivamente fra 50 e 6000 Hz; dei due microfoni che escono 
dal cntrlMic1 e cntrlMic2 ne viene ricavato l'inviluppo d'ampiezza tramite
l'oggetto \textit{integrator}, poi la curva di questo inviluppo viene amplificata da 
un delay con Feedback, e ne viene estratta solamente la componente energetica 
presente fra i 0 e i 0.5Hz, limitandone il comportamento dell'oscillazione in quel range.
Più in basso, nel SIGNAL FLOW 2a, i stessi microfoni (mic1 e mic2) sempre
limitati in banda fra 50 e 6000 Hz, vengono attenuati in ampiezza dal segnale
proveniente dal cntrlMic1 e cntrlMic2, realizzando nella sostanza una 
versione più sofisticata del controbilanciamento del meccanismo LAR.
La parte restante: ovvero, la generazione del segnale directLevel e la 
conseguente attenuazione dei segnali audio in uscita (sig1 e sig2),
consiste in un secondo controbilanciamento ricavato da un processo di granulazione
attivo alla fine del sistema, finché il granulatore non produce segnale in output 
l'ampiezza dei segnali audio diretti in uscita dal sistema rimane inalterata. \\
L'inviluppo d'ampiezza ricavato tramite l'oggetto \textit{integrator}, in partitura 
viene richiesto come segue:

\begin{center}
    \vspace{0.5cm}
    \textit{integrator = returns the average absolute value over a specific 
    time frame (one may use RMS measures, instead, or other amplitude-following methods); 
    output range is [0, 1]}
    \vspace{0.5cm}
\end{center}

dove sostanzialmente, viene calcolato il valore assoluto di una specifica finestra temporale. 
Viene indicato all'esecutore di poter utilizzare diversi metodi per poter calcolare il valore, come ad esempio l'RMS, 
tuttavia da analisi dirette secondo l'implementazione del compositore del sistema sul 
linguaggio di programmazione proprietario KYMA, i filtri utilizzati risultano essere di tipo IIR,
così da avere nella sostanza una risposta più lenta rispetto ad altre implementazioni come ad esempio 
quella del calcolo a media mobile.  
La risposta del filtro nell'implementazione originale è di tipo \textit{tau} 
e l'inviluppo è assoluto. \\
Nella loro interezza i meccanismi di cntrlMic1 e cntrlMic2, e la conseguente
attenuazione del segnale diretto provenente da mic1 e mic2 
possono essere dunque implementati in Faust come segue:
\clearpage 

\vspace{0.5cm}
\begin{lstlisting}
// import faust standard library
import("stdfaust.lib");
// import audible ecosystemics objects library
import("aelibrary.lib");
    
LARmechanismAE2(mic1, mic2) = sig1, sig2
with{
    // from Signal Flow 1b
    cntrlMic(x) = x : HPButterworthN(1, 50) : LPButterworthN(1, 6000) : 
            integrator(.01) : delayfb(.01, .995) : LPButterworthN(5, .5);
    
    cntrlMic1 = mic1 : cntrlMic;
    cntrlMic2 = mic2 : cntrlMic;
    
    // from Signal Flow 2a
    micIN1 = mic1 : HPButterworthN(1, 50) : LPButterworthN(1, 6000) * 
        (1 - cntrlMic1);
    micIN2 = mic2 : HPButterworthN(1, 50) : LPButterworthN(1, 6000) * 
        (1 - cntrlMic2);
    
    // in the full system this this is a secondary counterbalance
    directLevel = 1;
    sig1 = micIN1 * directLevel;
    sig2 = micIN2 * directLevel;
};
    
process = LARmechanismAE2;
\end{lstlisting}

Gli algoritmi degli oggetti richiamati in questa implementazione sono 
provenienti dalla libreria scritta per questo brano "aelibrary.lib" 
che è consultabile nel primo appendice.
Questo codice che rappresenta il cuore del sistema,
permette grazie al meccanismo adattivo una condizione favorevole
per mantenere il Larsen in uno stato stazionario, consentendo al sistema
di fatto una serie di nuove possibilità di elaborazione e trasformazione del suono.

\clearpage
\subsection{Trasformazioni del suono}
\label{sec:Trasformazioni del suono}

l’Ecosistemico Udibile n.2, studio sul Feedback, è costituito
da due principali meccanismi di Feedback, uno nello spazio acustico reale,
e uno interno al sistema. \\
Il suono proveniente dal meccanismo LAR appena discusso nella precedente sezione,
passa per una serie di elaborazioni numeriche del segnale prima di essere restituito
e diffuso dagli altoparlanti nello spazio acustico reale. Queste elaborazioni
che vengono poi nuovamente intercettate dal microfono e riportate nel Sistema,
consentono una continua e incessante trasformazione del suono, che permette
di cambiarne la morfologia ad ogni istante di tempo e in modo continuo, 
permettendo al sistema di avere "vita propria" ascoltandosi tramite l'ambiente
anche senza il contributo apportato da un esecutore. \\
Le elaborazioni del suono interne al sistema sono a loro volta di due tipologie,
\textit{sample read} e \textit{granular sampling}. \\
L'implementazione di questi due oggetti viene richiesta in partitura come segue:

\begin{center}
    \vspace{0.5cm}
    \textit{sample write = write samples into a memory buffer, in cyclical fashion (wrap-around)} \\
    \textit{sample read = read samples off the memory buffer, with controls over frequency shift ratios and actual buffer segment being read} \\
    \textit{granular sampling = read sample sequences off subsequent buffer memory chunks, and envelopes the signal chunk with a pseudo-Gaussian envelope curve; the particular
    implementation should allow for time-stretching (slower memory pointer increments at grain level), as well as for "grain density" controls and slight random deviations ("jitter") on
    grain parameters; no frequency shift necessary} \\
    \vspace{0.5cm}
\end{center}

dove \textit{sample write} rappresenta una tabella (buffer o dispositivo di memoria) 
che viene riscritta in modo continuo da un puntatore alla memoria;
ciclicamente, e in tutta la sua interezza.
La grandezza di questa tabella è data da una delle quattro variabili definite 
in partitura utilizzate per inizializzare il sistema, \textit{var1} che definito in questo contesto sta ad indicare
la grandezza della tabella in secondi.
Le 4 variabili definite per inizializzare il sistema ad ogni performance sono le seguenti:

\begin{center}
    \vspace{0.5cm}
    \textit{var1 = distance (in meters) between the two farthest removed loudspeakers on the left-right axis.} \\
    \textit{var2 = rough estimate of the center frequency in the spectrum of the room’s background noise (spectral centroid): to evaluate at rehearsal time, in a situation of "silence".} \\
    \textit{var3 = subjective estimate of how the room revereberance, valued between 0 ("no reverb") and 1 (“very long reverb”).} \\
    \textit{var4 = distance (in meters) between the two farthest removed loudspeakers on the front-rear axis.}
    \vspace{0.5cm}
\end{center}

queste 4 variabili hanno un ruolo molto importante, poiché non solo vanno a determinare la velocità del
comportamento del sistema, ma ne vanno a determinare la sua sensibilità rispetto all'ambiente in cui viene 
eseguito, come ad esempio nel caso del \textit{sample read} la velocità di lettura dalla tabella \textit{sample write} 
e la porzione di tabella che ne viene letta. \\
I \textit{sample read} sono 5 indicati in partitura nella configurazione che segue.

\begin{center}
\includegraphics[width=14cm]{figures/SAMPLERSfeedbackstudy2017.pdf} \\
{Estratto dalla partitura di \textit{Audible Ecosystemics n.2/ Feedback Study (2003)} \\
revisione del 2017 - Agostino Di Scipio} \\ 
\vspace{0.5cm}
\end{center}

Il suono del Larsen proveniente da due microfoni e stabilizzato con il meccanismo LAR come 
illustrato in precedenza, viene a sommarsi per poi venire moltiplicato per un oscillatore a bassa frequenza
che ne modifica l'inviluppo d'ampiezza. Il risultato di questa operazione viene scritto in memoria
nella tabella \textit{sample write} e viene letto dai 5 campionatori distribuiti In
questa sezione del sistema. \\
I parametri esterni e tempo varianti che utilizza il campionatore sono: \textit{ratio} e \textit{mem chunk},
che corrispondono rispettivamente alla velocità di lettura e al frammento di buffer che viene letto
dalla tabella \textit{sample write}.
A seguito di una conversazione con il compositore è risultato essere fondamentale che
il campionatore aggiorni i valori di \textit{ratio} ad ogni campione, permettendo modulazioni
di frequenza derivate dal cambiamento delle velocità di lettura,
che si riflettono in cambiamenti di frequenza del Larsen registrato nel \textit{sample write}.
i \textit{ratio} vengono indicati in partitura con delle espressioni diverse per ognuno
dei 5 \textit{sample read}, ad esempio: \textit{ratio: (var2 + (diffHL * 1000)) / 261}, dove 

\begin{center}
    \vspace{0.5cm}
    \textit{ratio value sets the sample-read rate as a function of var2 and the control signal diffHL: ratio = 1 causes no rate
    change, hence no frequency shift of the sampled signal; values major of 1 determine shifts to higher frequencies; values
    between 0 and 1 determine shifts to lower frequencies}
    \vspace{0.5cm}
\end{center}

e dove \textit{diffHL} è un segnale
di controllo variabile in un range fra 0 e 1, che a run time a microfoni chiusi parte da una costante di 0.5.
La lettura del \textit{ratio} è sensibile rispetto all'ambiente, poiché il valore di \textit{diffHL}
che ne determina le modulazioni di frequenza, viene ricavato da una stima della centroide spettale 
nei segnali di controllo tramite una differenza fra filtri Highpass e Lowpass 
a cui viene passata come frequenza di taglio la variabile \textit{var2}.
Altrettanto sostanziale nel contribuire alle trasformazioni del Larsen 
è il comportamento del \textit{mem chunk}, che deve permettere una lettura
di frammenti di memoria che può andare dal silenzio: \textit{mem chunk: 0, 0} all'intera
lunghezza del buffer \textit{mem chunk: 0, 1}.
Anche in questo caso la maggiorparte dei campionatori implementa un \textit{mem chunk} 
determinato dai segnali di controllo variabili in un range fra 0 e 1, 
segnali di controllo che a run time e a microfoni chiusi 
partono da costanti distribuite di 0 e 1.
Per questo motivo è molto importante che il comportamento del \textit{mem chunk} 
possa tollerare soglie ai due estremi.
In partitura altre importanti indicazioni sul campionatore che riguardano 
lettura e scrittura sono le seguenti:

\begin{center}
    \vspace{0.5cm}
    \textit{all sample-read processes 
    should include short fade-in and fade-out
    ramps when sampling pointers wrap around the buffer, to avoid
    discontinuities \\
    more discontinuities may occur because the buffer is being written
    as it is also being read: these can be avoided by various means of
    one's own design}
    \vspace{0.5cm}
\end{center}

In Faust non è possibile separare la scrittura e la lettura da un buffer,
difatti esistono solo 2 tipi di oggetti per questo scopo:
la primitiva \textit{rdtable}, che può essere utilizzata per leggere da una tabella di sola lettura 
(predefinita prima del tempo di compilazione),
e la primitiva \textit{rwtable}, che può essere utilizzata per implementare una tabella di lettura/scrittura. 
Quest'ultima prende in input un segnale che può essere scritto nella tabella utilizzando un indice di scrittura e 
letto utilizzando un secondo indice destinato alla lettura.
Per implementare in Faust l'architettura del \textit{sample write} richiesta in partitura, 
si possono sincronizzare gli indici di scrittura di tutte le tabelle con un unico 
segnale (rampa) destinato a questo scopo e utilizzato dai \textit{sample read} e dai \textit{granular sampling} 
impostando una grandezza comune, che risulta essere in questo caso una costante espressa in \textit{var1} in secondi, 
condivisa da tutti i buffer in questione.
In Faust la grandezza della tabella deve essere espressa in campioni, per ottenere quindi
una grandezza espressa in secondi, bisogna dunque moltiplicare il numero dei secondi desiderati per
la frequenza di campionamento, definite entrambi come una costante (predefinita prima del tempo di compilazione).
Infine per completare le operazioni come richiesto nella partitura, si può sostituire l'oggetto \textit{rwtable}
con una mandata (send) del segnale destinato ad entrare in feedback ai \textit{sample read} ed in
feedforward al \textit{granular sampling}.
A seguito una implementazione del campionatore come richiesto in partitura.
\clearpage 

\vspace{0.5cm}
\begin{lstlisting}
// import faust standard library
import("stdfaust.lib");
// hard-coded: change this to match your samplerate
SampleRate = 44100;

sampler(lengthSec, memChunk, ratio, x) = 
    it.frwtable(3, bufferLen, .0, writePtr, x, readPtr) * window
with {
    bufferLen = lengthSec * SampleRate;
    writePtr = ba.period(bufferLen);
    grainLen = max(1, ba.if(writePtr > memChunk * bufferLen, 
        memChunk * bufferLen, 1));
    readPtr = y
        letrec {
            'y = (ratio + y) % grainLen;
        };
    window = min(1, abs(((readPtr + grainLen / 2) % grainLen) - 
        grainLen / 2) / 400);
};

process = sampler(4, hslider("memChunk", 0.05, 0, 1, .001), 
    hslider("ratio", 10, .1, 10, .001), os.osc(200)) <: _, _;
\end{lstlisting}

In questo campionatore, il \textit{ratio} è continuamente variabile, 
rendendo possibili le modulazioni di frequenza desiderate. 
Mentre il \textit{mem chunk} ricomincia la sua lettura del buffer dall'inizio 
ogni volta che la rampa di lettura incontra il valore del modulo determinato dalla sua variabile.
Per risolvere le discontinuità come richiesto in partitura, 
è necessario utilizzare un design personalizzato,
e nel design di questo campionatore, 
per evitare clicks ricorrenti ogni volta che la lettura del campionatore incrocia la sua scrittura, 
(visto che la lettura del \textit{mem chunk} parte sempre da inizio buffer)
la lettura parte solo quando la scrittura è di dimensioni maggiori rispetto al frammento \textit{mem chunk} 
che dovrebbe essere letto.
Infine, per evitare discontinuità ogni volta che la lettura ricomincia a leggere il buffer da 0, 
è stata utilizzata una finestratura trapezoidale con un fade-in e fade-out 
per un valore di campioni costante.