%!TEX encoding = UTF-8 Unicode
%!TEX root = ../thesisCASes.tex

%% TITOLO

\section{Conclusioni}
\label{Conclusioni}

In queste conclusioni, desidero approfondire alcune importanti questioni 
emerse sia dalle nuove esperienze fatte durante il corso dello studio di questi brani, 
sia dalle informazioni acquisite durante la stesura della mia tesi.

In primo luogo, va sottolineata la complessità nel ricostruire 
l'impatto delle teorie dei sistemi complessi e delle teorie cibernetiche 
sulla produzione musicale dei compositori. 
L'argomento dei sistemi complessi in musica è stato infatti trattato da ogni 
compositore in maniera soggettiva, 
sia in termini di interpretazione che di applicazione. 
Abbiamo visto come alcuni compositori abbiano preferito seguire delle strade più formali 
nell'applicazione di queste teorie, 
mentre per altri è stato fondamentale applicare direttamente queste 
teorie nella creazione stessa di macchine e partiture operative che riflettano 
la propria idea musicale nella generazione stessa del suono.
Questo secondo scenario è quello a cui mi sono riferito 
nel corso della tesi attraverso lo studio e l'implementazione
di alcuni dei meccanismi fondamentali presenti nelle
opere di Agostino Di Scipio e Dario Sanfilippo.

In aggiunta alle considerazioni precedenti, da un punto di vista storico, 
risulta complesso risalire ad una singola storia della cibernetica nella musica elettroacustica 
che non sia influenzata da fattori geografici. 
Ogni continente o addirittura ogni nazione ha avuto i propri pionieri 
che hanno operato con diversi metodi nell'ambito dei sistemi complessi nella musica elettroacustica. 
Pertanto da questo emerge come sia una sfida attuale, e una responsabilità della storia della musica elettroacustica, 
documentare questa storia in maniera esaustiva e non essere indifferenti di fronte a questo scenario.

In secondo luogo, lo studio dei meccanismi presenti nelle opere di Agostino Di Scipio e Dario Sanfilippo 
ha messo in luce l'importanza di adottare una posizione ontologica ed epistemologica 
quando si lavora con questo tipo di sistemi complessi. 
La scelta di analizzare determinati dati all'interno dei propri sistemi 
implica una posizione riguardo alla natura dei dati stessi. 
Ciò significa che un compositore potrebbe ritenere importanti informazioni che 
per un altro compositore potrebbero essere meno significative, 
e designare una struttura sistemica che, 
negli intenti e negli scopi di quest'ultimo, sia meno adeguata. 

Come emerge dagli argomenti trattati, quindi, ad ogni compositore dei sistemi complessi in musica 
viene richiesto in qualche modo di essere alla stregua di un "architetto" o "piccolo scienziato" dei propri sistemi,
per essere in grado di guidare l'intero processo creativo 
attraverso la propria esperienza personale e la propria pratica, 
interpretando e costruendo quei fenomeni e quelle relazioni che gli interessa emergano 
all'interno propri sistemi e brani.
Per questo motivo, il ruolo del compositore/esecutore dei suoi sistemi 
in tal senso diviene parte integrante di queste opere.

In luce rispetto a quanto appreso, 
sono riuscito a portare a termine tramite la composizione di RITI, un primo modello personale 
di un sistema complesso per il live electronics,
in grado di mostrare dinamiche non lineari sulla "soglia del caos" e una certa complessità.
I miei obiettivi futuri per questo lavoro includono lo studio di nuovi 
meccanismi che possa integrare per migliorare il sistema di pari passo alla mia crescente esperienza nell'ambito,
e migliorare di conseguenza la quantità e la varietà di risultati che possono essere prodotti.
Al contempo, spero di creare una documentazione crescente e abbondante su questo sistema 
il più possibile "a basso livello", e che sia il meno dipendente possibile da tecnologie esistenti,
anche questa è una importante sfida della musica elettroacustica nell'ottica 
di una sostenibilità.

In conclusione, i miei obiettivi futuri riguardano il concentrarmi 
sulla creazione di metodi e oggetti di analisi che informino cognitivamente 
i sistemi che sviluppo secondo quei dei processi che ritengo più interessanti nella mia esperienza personale. 
Nel tempo, intendo portare avanti questi studi per creare la mia personale letteratura, 
contribuendo in modo costruttivo e compositivo alla letteratura già esistente,
e mirando a migliorare costantemente la possibilità 
che i fenomeni emergenti e i sistemi che compongono
possano sempre di più riflettere e rappresentare le mie volontà artistiche, 
nel desiderio di affinare le mie pratiche per migliorare le capacità 
dei sistemi nel creare un'armonia tra il suono e la complessità dei processi che lo generano 
che possa chiamare come "la mia musica".