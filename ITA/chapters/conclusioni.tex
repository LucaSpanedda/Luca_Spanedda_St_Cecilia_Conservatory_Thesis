%!TEX encoding = UTF-8 Unicode
%!TEX root = ../thesisCASes.tex

%% TITOLO

\section{Conclusioni}
\label{Conclusioni}

\todo[inline,color=green]{le conclusioni, insieme all'abstract, formano il confine del tuo pensiero e del tuo lavoro. non puoi limitarti a scrivere due righe, come se fosse il resoconto di una gita in montagna.}

In conclusione, 
abbiamo avuto modo di introdurre l'argomento delle cibernetiche in Musica,
parlando di come l'esigenza di un pensiero sistemico 
abbia pian piano fatto breccia nell'operato
di alcuni dei più importanti compositori del XX Secolo. 
Arrivando fino ad osservare e studiare il realizzarsi di questa condizione
attraverso l'operato di Dario Sanfilippo e da Agostino Di Scipio, 
di cui abbiamo approfondito alcuni dei principali meccanismi presenti nei loro lavori compositivi,
nel desiderio di creare attraverso questi modelli
una strada da seguire per la composizione dei sistemi complessi adattivi per la
performance musicale in live electronics.
Ed Infine mettendo in luce l'importanza di una prospettiva 
ontologica ed epistemeologica nella composizione musicale sistemica.